% ****************************************************************************************************
% classicthesis-config.tex 
% formerly known as loadpackages.sty, classicthesis-ldpkg.sty, and classicthesis-preamble.sty 
% Use it at the beginning of your ClassicThesis.tex, or as a LaTeX Preamble 
% in your ClassicThesis.{tex,lyx} with % ****************************************************************************************************
% classicthesis-config.tex 
% formerly known as loadpackages.sty, classicthesis-ldpkg.sty, and classicthesis-preamble.sty 
% Use it at the beginning of your ClassicThesis.tex, or as a LaTeX Preamble 
% in your ClassicThesis.{tex,lyx} with % ****************************************************************************************************
% classicthesis-config.tex 
% formerly known as loadpackages.sty, classicthesis-ldpkg.sty, and classicthesis-preamble.sty 
% Use it at the beginning of your ClassicThesis.tex, or as a LaTeX Preamble 
% in your ClassicThesis.{tex,lyx} with % ****************************************************************************************************
% classicthesis-config.tex 
% formerly known as loadpackages.sty, classicthesis-ldpkg.sty, and classicthesis-preamble.sty 
% Use it at the beginning of your ClassicThesis.tex, or as a LaTeX Preamble 
% in your ClassicThesis.{tex,lyx} with \input{classicthesis-config}
% ****************************************************************************************************  
% If you like the classicthesis, then I would appreciate a postcard. 
% My address can be found in the file ClassicThesis.pdf. A collection 
% of the postcards I received so far is available online at 
% http://postcards.miede.de
% ****************************************************************************************************

% ****************************************************************************************************
% 1. Configure classicthesis for your needs here, e.g., remove "drafting" below 
% in order to deactivate the time-stamp on the pages
% ****************************************************************************************************
\PassOptionsToPackage{eulerchapternumbers,listings,%drafting
				 pdfspacing,%floatperchapter,%linedheaders,%
				 subfig,beramono,eulermath,parts}{classicthesis}										
% ********************************************************************
% Available options for classicthesis.sty 
% (see ClassicThesis.pdf for more information):
% drafting
% parts nochapters linedheaders
% eulerchapternumbers beramono eulermath pdfspacing minionprospacing
% tocaligned dottedtoc manychapters
% listings floatperchapter subfig
% ********************************************************************

% ********************************************************************
% Triggers for this config
% ******************************************************************** 
\usepackage{ifthen}
\newboolean{enable-backrefs} % enable backrefs in the bibliography
\setboolean{enable-backrefs}{false} % true false
% ****************************************************************************************************


% ****************************************************************************************************
% 2. Personal data and user ad-hoc commands
% ****************************************************************************************************
\newcommand{\myTitle}{Interpreting natural language sentences into machine readable instructions\xspace}
%\newcommand{\mySubtitle}{An Homage to The Elements of Typographic Style\xspace}
\newcommand{\myDegree}{Master of Science in Computer Science\xspace}
\newcommand{\myName}{Martin Agfjord\xspace}
\newcommand{\mySupervisor}{Krasimir Angelov\xspace}
\newcommand{\myDepartment}{Department of Computer Science and Engineering\xspace}
\newcommand{\myUni}{University of Gothenburg\xspace}
\newcommand{\myLocation}{Gothenburg\xspace}
\newcommand{\myTime}{August 2014\xspace}
\newcommand{\myVersion}{version 1.0\xspace}

% ********************************************************************
% Setup, finetuning, and useful commands
% ********************************************************************
\newcounter{dummy} % necessary for correct hyperlinks (to index, bib, etc.)
\newlength{\abcd} % for ab..z string length calculation
\providecommand{\mLyX}{L\kern-.1667em\lower.25em\hbox{Y}\kern-.125emX\@}
\newcommand{\ie}{i.\,e.}
\newcommand{\Ie}{I.\,e.}
\newcommand{\eg}{e.\,g.}
\newcommand{\Eg}{E.\,g.} 
% ****************************************************************************************************


% ****************************************************************************************************
% 3. Loading some handy packages
% ****************************************************************************************************
% ******************************************************************** 
% Packages with options that might require adjustments
% ******************************************************************** 
%\PassOptionsToPackage{utf8}{inputenc}	% latin9 (ISO-8859-9) = latin1+"Euro sign"
\usepackage[utf8]{inputenc}	
%\PassOptionsToPackage{swedish}{babel}   % change this to your language(s)
% Spanish languages need extra options in order to work with this template
%\PassOptionsToPackage{spanish,es-lcroman}{babel}
 \usepackage[swedish]{babel}

\PassOptionsToPackage{square,numbers}{natbib}
\usepackage{natbib}
\PassOptionsToPackage{fleqn}{amsmath}		% math environments and more by the AMS 
 \usepackage{amsmath}

%\newunicodechar{00E4}{ä}
% ******************************************************************** 
% General useful packages
% ******************************************************************** 
%\PassOptionsToPackage{T1}{fontenc} % T2A for cyrillics
	\usepackage[T1]{fontenc}
\usepackage{textcomp} % fix warning with missing font shapes 
\usepackage{xspace} % to get the spacing after macros right  
\usepackage{mparhack} % get marginpar right
\usepackage{fixltx2e} % fixes some LaTeX stuff 
\PassOptionsToPackage{printonlyused,smaller}{acronym}
	\usepackage{acronym} % nice macros for handling all acronyms in the thesis
%\renewcommand*{\acsfont}[1]{\textssc{#1}} % for MinionPro
\renewcommand{\bflabel}[1]{{#1}\hfill} % fix the list of acronyms
% ****************************************************************************************************


% ****************************************************************************************************
% 4. Setup floats: tables, (sub)figures, and captions
% ****************************************************************************************************
\usepackage{tabularx} % better tables
	\setlength{\extrarowheight}{3pt} % increase table row height
\newcommand{\tableheadline}[1]{\multicolumn{1}{c}{\spacedlowsmallcaps{#1}}}
\newcommand{\myfloatalign}{\centering} % to be used with each float for alignment
\usepackage{caption}
\captionsetup{format=hang,font=small}
\usepackage{subfig}  
% ****************************************************************************************************


% ****************************************************************************************************
% 5. Setup code listings
% ****************************************************************************************************
\usepackage{listings} 
%\lstset{emph={trueIndex,root},emphstyle=\color{BlueViolet}}%\underbar} % for special keywords
\lstset{language=[LaTeX]Tex,%C++,
    keywordstyle=\color{RoyalBlue},%\bfseries,
    basicstyle=\small\ttfamily,
    %identifierstyle=\color{NavyBlue},
    commentstyle=\color{Green}\ttfamily,
    stringstyle=\rmfamily,
    numbers=none,%left,%
    numberstyle=\scriptsize,%\tiny
    stepnumber=5,
    numbersep=8pt,
    showstringspaces=false,
    breaklines=true,
    frameround=ftff,
    frame=single,
    belowcaptionskip=.75\baselineskip
    %frame=L
} 
% ****************************************************************************************************    		   


% ****************************************************************************************************
% 6. PDFLaTeX, hyperreferences and citation backreferences
% ****************************************************************************************************
% ********************************************************************
% Using PDFLaTeX
% ********************************************************************
\PassOptionsToPackage{pdftex,hyperfootnotes=false,pdfpagelabels}{hyperref}
	\usepackage{hyperref}  % backref linktocpage pagebackref
\pdfcompresslevel=9
\pdfadjustspacing=1 
\PassOptionsToPackage{pdftex}{graphicx}
	\usepackage{graphicx} 

% ********************************************************************
% Setup the style of the backrefs from the bibliography
% (translate the options to any language you use)
% ********************************************************************
\newcommand{\backrefnotcitedstring}{\relax}%(Not cited.)
\newcommand{\backrefcitedsinglestring}[1]{(Cited on page~#1.)}
\newcommand{\backrefcitedmultistring}[1]{(Cited on pages~#1.)}
\ifthenelse{\boolean{enable-backrefs}}%
{%
		\PassOptionsToPackage{hyperpageref}{backref}
		\usepackage{backref} % to be loaded after hyperref package 
		   \renewcommand{\backreftwosep}{ and~} % separate 2 pages
		   \renewcommand{\backreflastsep}{, and~} % separate last of longer list
		   \renewcommand*{\backref}[1]{}  % disable standard
		   \renewcommand*{\backrefalt}[4]{% detailed backref
		      \ifcase #1 %
		         \backrefnotcitedstring%
		      \or%
		         \backrefcitedsinglestring{#2}%
		      \else%
		         \backrefcitedmultistring{#2}%
		      \fi}%
}{\relax}    

% ********************************************************************
% Hyperreferences
% ********************************************************************
\hypersetup{%
    %draft,	% = no hyperlinking at all (useful in b/w printouts)
    colorlinks=true, linktocpage=true, pdfstartpage=3, pdfstartview=FitV,%
    % uncomment the following line if you want to have black links (e.g., for printing)
    %colorlinks=false, linktocpage=false, pdfborder={0 0 0}, pdfstartpage=3, pdfstartview=FitV,% 
    breaklinks=true, pdfpagemode=UseNone, pageanchor=true, pdfpagemode=UseOutlines,%
    plainpages=false, bookmarksnumbered, bookmarksopen=true, bookmarksopenlevel=1,%
    hypertexnames=true, pdfhighlight=/O,%nesting=true,%frenchlinks,%
    urlcolor=webbrown, linkcolor=RoyalBlue, citecolor=webgreen, %pagecolor=RoyalBlue,%
    %urlcolor=Black, linkcolor=Black, citecolor=Black, %pagecolor=Black,%
    pdftitle={\myTitle},%
    pdfauthor={\textcopyright\ \myName, \myUni, \myFaculty},%
    pdfsubject={},%
    pdfkeywords={},%
    pdfcreator={pdfLaTeX},%
    pdfproducer={LaTeX with hyperref and classicthesis}%
}   
%\hypersetup{tex4ht}


% ********************************************************************
% Setup autoreferences
% ********************************************************************
% There are some issues regarding autorefnames
% http://www.ureader.de/msg/136221647.aspx
% http://www.tex.ac.uk/cgi-bin/texfaq2html?label=latexwords
% you have to redefine the makros for the 
% language you use, e.g., american, ngerman
% (as chosen when loading babel/AtBeginDocument)
% ********************************************************************
\makeatletter
\@ifpackageloaded{babel}%
    {%
       \addto\extrasamerican{%
					\renewcommand*{\figureautorefname}{Figure}%
					\renewcommand*{\tableautorefname}{Table}%
					\renewcommand*{\partautorefname}{Part}%
					\renewcommand*{\chapterautorefname}{Chapter}%
					\renewcommand*{\sectionautorefname}{Section}%
					\renewcommand*{\subsectionautorefname}{Section}%
					\renewcommand*{\subsubsectionautorefname}{Section}% 	
				}%
       \addto\extrasngerman{% 
					\renewcommand*{\paragraphautorefname}{Absatz}%
					\renewcommand*{\subparagraphautorefname}{Unterabsatz}%
					\renewcommand*{\footnoteautorefname}{Fu\"snote}%
					\renewcommand*{\FancyVerbLineautorefname}{Zeile}%
					\renewcommand*{\theoremautorefname}{Theorem}%
					\renewcommand*{\appendixautorefname}{Anhang}%
					\renewcommand*{\equationautorefname}{Gleichung}%        
					\renewcommand*{\itemautorefname}{Punkt}%
				}%	
			% Fix to getting autorefs for subfigures right (thanks to Belinda Vogt for changing the definition)
			\providecommand{\subfigureautorefname}{\figureautorefname}%  			
    }{\relax}
\makeatother


% ****************************************************************************************************
% 7. Last calls before the bar closes
% ****************************************************************************************************
% ********************************************************************
% Development Stuff
% ********************************************************************
\listfiles
%\PassOptionsToPackage{l2tabu,orthodox,abort}{nag}
%	\usepackage{nag}
%\PassOptionsToPackage{warning, all}{onlyamsmath}
%	\usepackage{onlyamsmath}

% ********************************************************************
% Last, but not least...
% ********************************************************************
\usepackage{classicthesis} 
% ****************************************************************************************************


% ****************************************************************************************************
% 8. Further adjustments (experimental)
% ****************************************************************************************************
% ********************************************************************
% Changing the text area
% ********************************************************************
%\linespread{1.05} % a bit more for Palatino
%\areaset[current]{312pt}{761pt} % 686 (factor 2.2) + 33 head + 42 head \the\footskip
%\setlength{\marginparwidth}{7em}%
%\setlength{\marginparsep}{2em}%

% ********************************************************************
% Using different fonts
% ********************************************************************
%\usepackage[oldstylenums]{kpfonts} % oldstyle notextcomp
%\usepackage[osf]{libertine}
%\usepackage{hfoldsty} % Computer Modern with osf
%\usepackage[light,condensed,math]{iwona}
%\renewcommand{\sfdefault}{iwona}
%\usepackage{lmodern} % <-- no osf support :-(
%\usepackage[urw-garamond]{mathdesign} <-- no osf support :-(
% ****************************************************************************************************

% *********************************************************************
% Syntax highlighting
% *********************************************************************
% \usepackage{fontspec}
\usepackage{minted}

%
% Fix to support extended characters in minted with combination of pdfTeX
% http://tex.stackexchange.com/questions/19781/using-package-minted-with-non-english-characters
%

\makeatletter
\newcommand{\minted@write@detok}[1]{%
  \immediate\write\FV@OutFile{\detokenize{#1}}}%

% Blank page
\newcommand*{\blankpage}{%
  \begingroup
    \setlength{\topskip}{0pt}%
    \setlength{\parskip}{0pt}%
    \vspace*{\fill}%
    \centering
    \nointerlineskip % suppresses partial \baselineskip above text
    This page is intentionally left (almost) blank.\par
    \thispagestyle{empty}
    \setcounter{page}{}
    \vspace{\fill}%
  \endgroup
  % \showlists
}

\newcommand{\minted@FVB@VerbatimOut}[1]{%
  \@bsphack
  \begingroup
    \FV@UseKeyValues
    \FV@DefineWhiteSpace
    \def\FV@Space{\space}%
    \FV@DefineTabOut
    %\def\FV@ProcessLine{\immediate\write\FV@OutFile}% %Old, non-Unicode version
    \let\FV@ProcessLine\minted@write@detok %Patch for Unicode
    \immediate\openout\FV@OutFile #1\relax
    \let\FV@FontScanPrep\relax
%% DG/SR modification begin - May. 18, 1998 (to avoid problems with ligatures)
    \let\@noligs\relax
%% DG/SR modification end
    \FV@Scan}
    \let\FVB@VerbatimOut\minted@FVB@VerbatimOut

\renewcommand\minted@savecode[1]{
  \immediate\openout\minted@code\jobname.pyg
  \immediate\write\minted@code{\expandafter\detokenize\expandafter{#1}}%
  \immediate\closeout\minted@code}
\makeatother

\usepackage{tcolorbox}
\tcbuselibrary{skins}
\usetikzlibrary{shadings}

\usepackage{etoolbox}

% Code
\BeforeBeginEnvironment{code}{\begin{tcolorbox}[enhanced,
width=\linewidth,
enlarge top by=3pt,enlarge bottom by=3pt,
enlarge left by=3pt,enlarge right by=3pt,
frame hidden,boxrule=0pt,top=1mm,bottom=1mm,
colframe=green!30!black, colbacktitle=green!50!yellow,
coltitle=black, colback=bg,
borderline={.1pt}{-0.5pt}{gray, dashed, sharp corners}]}%
\AfterEndEnvironment{code}{\end{tcolorbox}}

% Terminal
\BeforeBeginEnvironment{terminal}{\begin{tcolorbox}[enhanced,
width=\linewidth,
enlarge top by=3pt,enlarge bottom by=3pt,
enlarge left by=3pt,enlarge right by=3pt,
frame hidden,boxrule=0pt,top=1mm,bottom=1mm,
colframe=green!30!black, colbacktitle=green!50!yellow,
coltitle=black, colback=bg,
borderline={.1pt}{-0.5pt}{gray, dashed, sharp corners}]}%
\AfterEndEnvironment{terminal}{\end{tcolorbox}}

% Java
\BeforeBeginEnvironment{java-code}{\begin{tcolorbox}[enhanced,
width=\linewidth,
enlarge top by=3pt,enlarge bottom by=3pt,
enlarge left by=3pt,enlarge right by=3pt,
frame hidden,boxrule=0pt,top=1mm,bottom=1mm,
colframe=green!30!black, colbacktitle=green!50!yellow,
coltitle=black, colback=bg,
borderline={.1pt}{-0.5pt}{gray, dashed, sharp corners}]}%
\AfterEndEnvironment{java-code}{\end{tcolorbox}}

% Js
\BeforeBeginEnvironment{js-code}{\begin{tcolorbox}[enhanced,
width=\linewidth,
enlarge top by=3pt,enlarge bottom by=3pt,
enlarge left by=3pt,enlarge right by=3pt,
frame hidden,boxrule=0pt,top=1mm,bottom=1mm,
colframe=green!30!black, colbacktitle=green!50!yellow,
coltitle=black, colback=bg,
borderline={.1pt}{-0.5pt}{gray, dashed, sharp corners}]}%
\AfterEndEnvironment{js-code}{\end{tcolorbox}}

% Json
\BeforeBeginEnvironment{json-normal}{\begin{tcolorbox}[enhanced,
width=\linewidth,
enlarge top by=3pt,enlarge bottom by=3pt,
enlarge left by=3pt,enlarge right by=3pt,
frame hidden,boxrule=0pt,top=1mm,bottom=1mm,
colframe=green!30!black, colbacktitle=green!50!yellow,
coltitle=black, colback=bg,
borderline={.1pt}{-0.5pt}{gray, dashed, sharp corners}]}%
\AfterEndEnvironment{json-normal}{\end{tcolorbox}}

% Json
\BeforeBeginEnvironment{json-small}{\begin{tcolorbox}[enhanced,
width=\linewidth,
enlarge top by=3pt,enlarge bottom by=3pt,
enlarge left by=3pt,enlarge right by=3pt,
frame hidden,boxrule=0pt,top=1mm,bottom=1mm,
colframe=green!30!black, colbacktitle=green!50!yellow,
coltitle=black, colback=bg,
borderline={.1pt}{-0.5pt}{gray, dashed, sharp corners}]}%
\AfterEndEnvironment{json-small}{\end{tcolorbox}}


% Plaintext
\BeforeBeginEnvironment{plaintext}{\begin{tcolorbox}[enhanced,
frame hidden,boxrule=0pt,top=1mm,bottom=0.5mm]}%
\AfterEndEnvironment{plaintext}{\end{tcolorbox}}%

\definecolor{bg}{HTML}{F2F2F2}
\newminted[code]{text}{mathescape=true,linenos=true,numbersep=5pt,fontsize=\footnotesize,framesep=10pt}
\newminted[java-code]{csharp}{linenos=true,numbersep=5pt,fontsize=\footnotesize,framesep=10pt}
\newminted[js-code]{js}{linenos=true,numbersep=5pt,fontsize=\footnotesize,framesep=10pt}
\newminted[json-normal]{js}{linenos=true,numbersep=5pt,fontsize=\footnotesize,framesep=10pt}
\newminted[json-small]{js}{linenos=true,numbersep=5pt,fontsize=\scriptsize,framesep=10pt}
\newminted[terminal]{text}{linenos=false,numbersep=5pt,fontsize=\footnotesize,framesep=10pt}
\newminted[plaintext]{text}{linenos=false,numbersep=5pt,fontsize=\footnotesize,framesep=10pt}
\newminted[bashcode2]{bash}{linenos=true, texcl=true}
\usepackage{scrhack} % fix warnings when using KOMA with listings package         
%\setsansfont{Calibri}
%\setmonofont{Consolas}
%\renewcommand{\familydefault}{\sfdefault}

\usepackage{etoolbox}

% Margin
\usepackage[left=1.5in,right=1.5in,top=1.5in,bottom=1.5in]{geometry}
\usepackage[framemethod=default]{mdframed}

%Exact positioning
\usepackage{float}

%Hyper-links
\usepackage{hyperref}

%Todo color
\usepackage{xcolor}
\newcommand\todo[1]{\textcolor{red}{#1}}

% Centering of graphics
\usepackage[export]{adjustbox}% http://ctan.org/pkg/adjustbox

%Hidden sections
\newcommand{\hiddensubsection}[1]{
    \stepcounter{subsection}
    \subsection*{\arabic{chapter}.\arabic{section}.\arabic{subsection}\hspace{1em}{#1}}
}

\newcommand{\hiddensubsubsection}[1]{
    \stepcounter{subsubsection}
    \subsubsection*{\arabic{chapter}.\arabic{section}.\arabic{subsection}\hspace{1em}{#1}}
}

\newcommand{\hiddensubsubsubsection}[1]{
    \stepcounter{subsubsubsection}
    \subsubsubsection*{\arabic{chapter}.\arabic{section}.\arabic{subsection}\hspace{1em}{#1}}
}

% Include pdfs
\usepackage{pdfpages}
% ****************************************************************************************************  
% If you like the classicthesis, then I would appreciate a postcard. 
% My address can be found in the file ClassicThesis.pdf. A collection 
% of the postcards I received so far is available online at 
% http://postcards.miede.de
% ****************************************************************************************************

% ****************************************************************************************************
% 1. Configure classicthesis for your needs here, e.g., remove "drafting" below 
% in order to deactivate the time-stamp on the pages
% ****************************************************************************************************
\PassOptionsToPackage{eulerchapternumbers,listings,%drafting
				 pdfspacing,%floatperchapter,%linedheaders,%
				 subfig,beramono,eulermath,parts}{classicthesis}										
% ********************************************************************
% Available options for classicthesis.sty 
% (see ClassicThesis.pdf for more information):
% drafting
% parts nochapters linedheaders
% eulerchapternumbers beramono eulermath pdfspacing minionprospacing
% tocaligned dottedtoc manychapters
% listings floatperchapter subfig
% ********************************************************************

% ********************************************************************
% Triggers for this config
% ******************************************************************** 
\usepackage{ifthen}
\newboolean{enable-backrefs} % enable backrefs in the bibliography
\setboolean{enable-backrefs}{false} % true false
% ****************************************************************************************************


% ****************************************************************************************************
% 2. Personal data and user ad-hoc commands
% ****************************************************************************************************
\newcommand{\myTitle}{Interpreting natural language sentences into machine readable instructions\xspace}
%\newcommand{\mySubtitle}{An Homage to The Elements of Typographic Style\xspace}
\newcommand{\myDegree}{Master of Science in Computer Science\xspace}
\newcommand{\myName}{Martin Agfjord\xspace}
\newcommand{\mySupervisor}{Krasimir Angelov\xspace}
\newcommand{\myDepartment}{Department of Computer Science and Engineering\xspace}
\newcommand{\myUni}{University of Gothenburg\xspace}
\newcommand{\myLocation}{Gothenburg\xspace}
\newcommand{\myTime}{August 2014\xspace}
\newcommand{\myVersion}{version 1.0\xspace}

% ********************************************************************
% Setup, finetuning, and useful commands
% ********************************************************************
\newcounter{dummy} % necessary for correct hyperlinks (to index, bib, etc.)
\newlength{\abcd} % for ab..z string length calculation
\providecommand{\mLyX}{L\kern-.1667em\lower.25em\hbox{Y}\kern-.125emX\@}
\newcommand{\ie}{i.\,e.}
\newcommand{\Ie}{I.\,e.}
\newcommand{\eg}{e.\,g.}
\newcommand{\Eg}{E.\,g.} 
% ****************************************************************************************************


% ****************************************************************************************************
% 3. Loading some handy packages
% ****************************************************************************************************
% ******************************************************************** 
% Packages with options that might require adjustments
% ******************************************************************** 
%\PassOptionsToPackage{utf8}{inputenc}	% latin9 (ISO-8859-9) = latin1+"Euro sign"
\usepackage[utf8]{inputenc}	
%\PassOptionsToPackage{swedish}{babel}   % change this to your language(s)
% Spanish languages need extra options in order to work with this template
%\PassOptionsToPackage{spanish,es-lcroman}{babel}
 \usepackage[swedish]{babel}

\PassOptionsToPackage{square,numbers}{natbib}
\usepackage{natbib}
\PassOptionsToPackage{fleqn}{amsmath}		% math environments and more by the AMS 
 \usepackage{amsmath}

%\newunicodechar{00E4}{ä}
% ******************************************************************** 
% General useful packages
% ******************************************************************** 
%\PassOptionsToPackage{T1}{fontenc} % T2A for cyrillics
	\usepackage[T1]{fontenc}
\usepackage{textcomp} % fix warning with missing font shapes 
\usepackage{xspace} % to get the spacing after macros right  
\usepackage{mparhack} % get marginpar right
\usepackage{fixltx2e} % fixes some LaTeX stuff 
\PassOptionsToPackage{printonlyused,smaller}{acronym}
	\usepackage{acronym} % nice macros for handling all acronyms in the thesis
%\renewcommand*{\acsfont}[1]{\textssc{#1}} % for MinionPro
\renewcommand{\bflabel}[1]{{#1}\hfill} % fix the list of acronyms
% ****************************************************************************************************


% ****************************************************************************************************
% 4. Setup floats: tables, (sub)figures, and captions
% ****************************************************************************************************
\usepackage{tabularx} % better tables
	\setlength{\extrarowheight}{3pt} % increase table row height
\newcommand{\tableheadline}[1]{\multicolumn{1}{c}{\spacedlowsmallcaps{#1}}}
\newcommand{\myfloatalign}{\centering} % to be used with each float for alignment
\usepackage{caption}
\captionsetup{format=hang,font=small}
\usepackage{subfig}  
% ****************************************************************************************************


% ****************************************************************************************************
% 5. Setup code listings
% ****************************************************************************************************
\usepackage{listings} 
%\lstset{emph={trueIndex,root},emphstyle=\color{BlueViolet}}%\underbar} % for special keywords
\lstset{language=[LaTeX]Tex,%C++,
    keywordstyle=\color{RoyalBlue},%\bfseries,
    basicstyle=\small\ttfamily,
    %identifierstyle=\color{NavyBlue},
    commentstyle=\color{Green}\ttfamily,
    stringstyle=\rmfamily,
    numbers=none,%left,%
    numberstyle=\scriptsize,%\tiny
    stepnumber=5,
    numbersep=8pt,
    showstringspaces=false,
    breaklines=true,
    frameround=ftff,
    frame=single,
    belowcaptionskip=.75\baselineskip
    %frame=L
} 
% ****************************************************************************************************    		   


% ****************************************************************************************************
% 6. PDFLaTeX, hyperreferences and citation backreferences
% ****************************************************************************************************
% ********************************************************************
% Using PDFLaTeX
% ********************************************************************
\PassOptionsToPackage{pdftex,hyperfootnotes=false,pdfpagelabels}{hyperref}
	\usepackage{hyperref}  % backref linktocpage pagebackref
\pdfcompresslevel=9
\pdfadjustspacing=1 
\PassOptionsToPackage{pdftex}{graphicx}
	\usepackage{graphicx} 

% ********************************************************************
% Setup the style of the backrefs from the bibliography
% (translate the options to any language you use)
% ********************************************************************
\newcommand{\backrefnotcitedstring}{\relax}%(Not cited.)
\newcommand{\backrefcitedsinglestring}[1]{(Cited on page~#1.)}
\newcommand{\backrefcitedmultistring}[1]{(Cited on pages~#1.)}
\ifthenelse{\boolean{enable-backrefs}}%
{%
		\PassOptionsToPackage{hyperpageref}{backref}
		\usepackage{backref} % to be loaded after hyperref package 
		   \renewcommand{\backreftwosep}{ and~} % separate 2 pages
		   \renewcommand{\backreflastsep}{, and~} % separate last of longer list
		   \renewcommand*{\backref}[1]{}  % disable standard
		   \renewcommand*{\backrefalt}[4]{% detailed backref
		      \ifcase #1 %
		         \backrefnotcitedstring%
		      \or%
		         \backrefcitedsinglestring{#2}%
		      \else%
		         \backrefcitedmultistring{#2}%
		      \fi}%
}{\relax}    

% ********************************************************************
% Hyperreferences
% ********************************************************************
\hypersetup{%
    %draft,	% = no hyperlinking at all (useful in b/w printouts)
    colorlinks=true, linktocpage=true, pdfstartpage=3, pdfstartview=FitV,%
    % uncomment the following line if you want to have black links (e.g., for printing)
    %colorlinks=false, linktocpage=false, pdfborder={0 0 0}, pdfstartpage=3, pdfstartview=FitV,% 
    breaklinks=true, pdfpagemode=UseNone, pageanchor=true, pdfpagemode=UseOutlines,%
    plainpages=false, bookmarksnumbered, bookmarksopen=true, bookmarksopenlevel=1,%
    hypertexnames=true, pdfhighlight=/O,%nesting=true,%frenchlinks,%
    urlcolor=webbrown, linkcolor=RoyalBlue, citecolor=webgreen, %pagecolor=RoyalBlue,%
    %urlcolor=Black, linkcolor=Black, citecolor=Black, %pagecolor=Black,%
    pdftitle={\myTitle},%
    pdfauthor={\textcopyright\ \myName, \myUni, \myFaculty},%
    pdfsubject={},%
    pdfkeywords={},%
    pdfcreator={pdfLaTeX},%
    pdfproducer={LaTeX with hyperref and classicthesis}%
}   
%\hypersetup{tex4ht}


% ********************************************************************
% Setup autoreferences
% ********************************************************************
% There are some issues regarding autorefnames
% http://www.ureader.de/msg/136221647.aspx
% http://www.tex.ac.uk/cgi-bin/texfaq2html?label=latexwords
% you have to redefine the makros for the 
% language you use, e.g., american, ngerman
% (as chosen when loading babel/AtBeginDocument)
% ********************************************************************
\makeatletter
\@ifpackageloaded{babel}%
    {%
       \addto\extrasamerican{%
					\renewcommand*{\figureautorefname}{Figure}%
					\renewcommand*{\tableautorefname}{Table}%
					\renewcommand*{\partautorefname}{Part}%
					\renewcommand*{\chapterautorefname}{Chapter}%
					\renewcommand*{\sectionautorefname}{Section}%
					\renewcommand*{\subsectionautorefname}{Section}%
					\renewcommand*{\subsubsectionautorefname}{Section}% 	
				}%
       \addto\extrasngerman{% 
					\renewcommand*{\paragraphautorefname}{Absatz}%
					\renewcommand*{\subparagraphautorefname}{Unterabsatz}%
					\renewcommand*{\footnoteautorefname}{Fu\"snote}%
					\renewcommand*{\FancyVerbLineautorefname}{Zeile}%
					\renewcommand*{\theoremautorefname}{Theorem}%
					\renewcommand*{\appendixautorefname}{Anhang}%
					\renewcommand*{\equationautorefname}{Gleichung}%        
					\renewcommand*{\itemautorefname}{Punkt}%
				}%	
			% Fix to getting autorefs for subfigures right (thanks to Belinda Vogt for changing the definition)
			\providecommand{\subfigureautorefname}{\figureautorefname}%  			
    }{\relax}
\makeatother


% ****************************************************************************************************
% 7. Last calls before the bar closes
% ****************************************************************************************************
% ********************************************************************
% Development Stuff
% ********************************************************************
\listfiles
%\PassOptionsToPackage{l2tabu,orthodox,abort}{nag}
%	\usepackage{nag}
%\PassOptionsToPackage{warning, all}{onlyamsmath}
%	\usepackage{onlyamsmath}

% ********************************************************************
% Last, but not least...
% ********************************************************************
\usepackage{classicthesis} 
% ****************************************************************************************************


% ****************************************************************************************************
% 8. Further adjustments (experimental)
% ****************************************************************************************************
% ********************************************************************
% Changing the text area
% ********************************************************************
%\linespread{1.05} % a bit more for Palatino
%\areaset[current]{312pt}{761pt} % 686 (factor 2.2) + 33 head + 42 head \the\footskip
%\setlength{\marginparwidth}{7em}%
%\setlength{\marginparsep}{2em}%

% ********************************************************************
% Using different fonts
% ********************************************************************
%\usepackage[oldstylenums]{kpfonts} % oldstyle notextcomp
%\usepackage[osf]{libertine}
%\usepackage{hfoldsty} % Computer Modern with osf
%\usepackage[light,condensed,math]{iwona}
%\renewcommand{\sfdefault}{iwona}
%\usepackage{lmodern} % <-- no osf support :-(
%\usepackage[urw-garamond]{mathdesign} <-- no osf support :-(
% ****************************************************************************************************

% *********************************************************************
% Syntax highlighting
% *********************************************************************
% \usepackage{fontspec}
\usepackage{minted}

%
% Fix to support extended characters in minted with combination of pdfTeX
% http://tex.stackexchange.com/questions/19781/using-package-minted-with-non-english-characters
%

\makeatletter
\newcommand{\minted@write@detok}[1]{%
  \immediate\write\FV@OutFile{\detokenize{#1}}}%

% Blank page
\newcommand*{\blankpage}{%
  \begingroup
    \setlength{\topskip}{0pt}%
    \setlength{\parskip}{0pt}%
    \vspace*{\fill}%
    \centering
    \nointerlineskip % suppresses partial \baselineskip above text
    This page is intentionally left (almost) blank.\par
    \thispagestyle{empty}
    \setcounter{page}{}
    \vspace{\fill}%
  \endgroup
  % \showlists
}

\newcommand{\minted@FVB@VerbatimOut}[1]{%
  \@bsphack
  \begingroup
    \FV@UseKeyValues
    \FV@DefineWhiteSpace
    \def\FV@Space{\space}%
    \FV@DefineTabOut
    %\def\FV@ProcessLine{\immediate\write\FV@OutFile}% %Old, non-Unicode version
    \let\FV@ProcessLine\minted@write@detok %Patch for Unicode
    \immediate\openout\FV@OutFile #1\relax
    \let\FV@FontScanPrep\relax
%% DG/SR modification begin - May. 18, 1998 (to avoid problems with ligatures)
    \let\@noligs\relax
%% DG/SR modification end
    \FV@Scan}
    \let\FVB@VerbatimOut\minted@FVB@VerbatimOut

\renewcommand\minted@savecode[1]{
  \immediate\openout\minted@code\jobname.pyg
  \immediate\write\minted@code{\expandafter\detokenize\expandafter{#1}}%
  \immediate\closeout\minted@code}
\makeatother

\usepackage{tcolorbox}
\tcbuselibrary{skins}
\usetikzlibrary{shadings}

\usepackage{etoolbox}

% Code
\BeforeBeginEnvironment{code}{\begin{tcolorbox}[enhanced,
width=\linewidth,
enlarge top by=3pt,enlarge bottom by=3pt,
enlarge left by=3pt,enlarge right by=3pt,
frame hidden,boxrule=0pt,top=1mm,bottom=1mm,
colframe=green!30!black, colbacktitle=green!50!yellow,
coltitle=black, colback=bg,
borderline={.1pt}{-0.5pt}{gray, dashed, sharp corners}]}%
\AfterEndEnvironment{code}{\end{tcolorbox}}

% Terminal
\BeforeBeginEnvironment{terminal}{\begin{tcolorbox}[enhanced,
width=\linewidth,
enlarge top by=3pt,enlarge bottom by=3pt,
enlarge left by=3pt,enlarge right by=3pt,
frame hidden,boxrule=0pt,top=1mm,bottom=1mm,
colframe=green!30!black, colbacktitle=green!50!yellow,
coltitle=black, colback=bg,
borderline={.1pt}{-0.5pt}{gray, dashed, sharp corners}]}%
\AfterEndEnvironment{terminal}{\end{tcolorbox}}

% Java
\BeforeBeginEnvironment{java-code}{\begin{tcolorbox}[enhanced,
width=\linewidth,
enlarge top by=3pt,enlarge bottom by=3pt,
enlarge left by=3pt,enlarge right by=3pt,
frame hidden,boxrule=0pt,top=1mm,bottom=1mm,
colframe=green!30!black, colbacktitle=green!50!yellow,
coltitle=black, colback=bg,
borderline={.1pt}{-0.5pt}{gray, dashed, sharp corners}]}%
\AfterEndEnvironment{java-code}{\end{tcolorbox}}

% Js
\BeforeBeginEnvironment{js-code}{\begin{tcolorbox}[enhanced,
width=\linewidth,
enlarge top by=3pt,enlarge bottom by=3pt,
enlarge left by=3pt,enlarge right by=3pt,
frame hidden,boxrule=0pt,top=1mm,bottom=1mm,
colframe=green!30!black, colbacktitle=green!50!yellow,
coltitle=black, colback=bg,
borderline={.1pt}{-0.5pt}{gray, dashed, sharp corners}]}%
\AfterEndEnvironment{js-code}{\end{tcolorbox}}

% Json
\BeforeBeginEnvironment{json-normal}{\begin{tcolorbox}[enhanced,
width=\linewidth,
enlarge top by=3pt,enlarge bottom by=3pt,
enlarge left by=3pt,enlarge right by=3pt,
frame hidden,boxrule=0pt,top=1mm,bottom=1mm,
colframe=green!30!black, colbacktitle=green!50!yellow,
coltitle=black, colback=bg,
borderline={.1pt}{-0.5pt}{gray, dashed, sharp corners}]}%
\AfterEndEnvironment{json-normal}{\end{tcolorbox}}

% Json
\BeforeBeginEnvironment{json-small}{\begin{tcolorbox}[enhanced,
width=\linewidth,
enlarge top by=3pt,enlarge bottom by=3pt,
enlarge left by=3pt,enlarge right by=3pt,
frame hidden,boxrule=0pt,top=1mm,bottom=1mm,
colframe=green!30!black, colbacktitle=green!50!yellow,
coltitle=black, colback=bg,
borderline={.1pt}{-0.5pt}{gray, dashed, sharp corners}]}%
\AfterEndEnvironment{json-small}{\end{tcolorbox}}


% Plaintext
\BeforeBeginEnvironment{plaintext}{\begin{tcolorbox}[enhanced,
frame hidden,boxrule=0pt,top=1mm,bottom=0.5mm]}%
\AfterEndEnvironment{plaintext}{\end{tcolorbox}}%

\definecolor{bg}{HTML}{F2F2F2}
\newminted[code]{text}{mathescape=true,linenos=true,numbersep=5pt,fontsize=\footnotesize,framesep=10pt}
\newminted[java-code]{csharp}{linenos=true,numbersep=5pt,fontsize=\footnotesize,framesep=10pt}
\newminted[js-code]{js}{linenos=true,numbersep=5pt,fontsize=\footnotesize,framesep=10pt}
\newminted[json-normal]{js}{linenos=true,numbersep=5pt,fontsize=\footnotesize,framesep=10pt}
\newminted[json-small]{js}{linenos=true,numbersep=5pt,fontsize=\scriptsize,framesep=10pt}
\newminted[terminal]{text}{linenos=false,numbersep=5pt,fontsize=\footnotesize,framesep=10pt}
\newminted[plaintext]{text}{linenos=false,numbersep=5pt,fontsize=\footnotesize,framesep=10pt}
\newminted[bashcode2]{bash}{linenos=true, texcl=true}
\usepackage{scrhack} % fix warnings when using KOMA with listings package         
%\setsansfont{Calibri}
%\setmonofont{Consolas}
%\renewcommand{\familydefault}{\sfdefault}

\usepackage{etoolbox}

% Margin
\usepackage[left=1.5in,right=1.5in,top=1.5in,bottom=1.5in]{geometry}
\usepackage[framemethod=default]{mdframed}

%Exact positioning
\usepackage{float}

%Hyper-links
\usepackage{hyperref}

%Todo color
\usepackage{xcolor}
\newcommand\todo[1]{\textcolor{red}{#1}}

% Centering of graphics
\usepackage[export]{adjustbox}% http://ctan.org/pkg/adjustbox

%Hidden sections
\newcommand{\hiddensubsection}[1]{
    \stepcounter{subsection}
    \subsection*{\arabic{chapter}.\arabic{section}.\arabic{subsection}\hspace{1em}{#1}}
}

\newcommand{\hiddensubsubsection}[1]{
    \stepcounter{subsubsection}
    \subsubsection*{\arabic{chapter}.\arabic{section}.\arabic{subsection}\hspace{1em}{#1}}
}

\newcommand{\hiddensubsubsubsection}[1]{
    \stepcounter{subsubsubsection}
    \subsubsubsection*{\arabic{chapter}.\arabic{section}.\arabic{subsection}\hspace{1em}{#1}}
}

% Include pdfs
\usepackage{pdfpages}
% ****************************************************************************************************  
% If you like the classicthesis, then I would appreciate a postcard. 
% My address can be found in the file ClassicThesis.pdf. A collection 
% of the postcards I received so far is available online at 
% http://postcards.miede.de
% ****************************************************************************************************

% ****************************************************************************************************
% 1. Configure classicthesis for your needs here, e.g., remove "drafting" below 
% in order to deactivate the time-stamp on the pages
% ****************************************************************************************************
\PassOptionsToPackage{eulerchapternumbers,listings,%drafting
				 pdfspacing,%floatperchapter,%linedheaders,%
				 subfig,beramono,eulermath,parts}{classicthesis}										
% ********************************************************************
% Available options for classicthesis.sty 
% (see ClassicThesis.pdf for more information):
% drafting
% parts nochapters linedheaders
% eulerchapternumbers beramono eulermath pdfspacing minionprospacing
% tocaligned dottedtoc manychapters
% listings floatperchapter subfig
% ********************************************************************

% ********************************************************************
% Triggers for this config
% ******************************************************************** 
\usepackage{ifthen}
\newboolean{enable-backrefs} % enable backrefs in the bibliography
\setboolean{enable-backrefs}{false} % true false
% ****************************************************************************************************


% ****************************************************************************************************
% 2. Personal data and user ad-hoc commands
% ****************************************************************************************************
\newcommand{\myTitle}{Interpreting natural language sentences into machine readable instructions\xspace}
%\newcommand{\mySubtitle}{An Homage to The Elements of Typographic Style\xspace}
\newcommand{\myDegree}{Master of Science in Computer Science\xspace}
\newcommand{\myName}{Martin Agfjord\xspace}
\newcommand{\mySupervisor}{Krasimir Angelov\xspace}
\newcommand{\myDepartment}{Department of Computer Science and Engineering\xspace}
\newcommand{\myUni}{University of Gothenburg\xspace}
\newcommand{\myLocation}{Gothenburg\xspace}
\newcommand{\myTime}{August 2014\xspace}
\newcommand{\myVersion}{version 1.0\xspace}

% ********************************************************************
% Setup, finetuning, and useful commands
% ********************************************************************
\newcounter{dummy} % necessary for correct hyperlinks (to index, bib, etc.)
\newlength{\abcd} % for ab..z string length calculation
\providecommand{\mLyX}{L\kern-.1667em\lower.25em\hbox{Y}\kern-.125emX\@}
\newcommand{\ie}{i.\,e.}
\newcommand{\Ie}{I.\,e.}
\newcommand{\eg}{e.\,g.}
\newcommand{\Eg}{E.\,g.} 
% ****************************************************************************************************


% ****************************************************************************************************
% 3. Loading some handy packages
% ****************************************************************************************************
% ******************************************************************** 
% Packages with options that might require adjustments
% ******************************************************************** 
%\PassOptionsToPackage{utf8}{inputenc}	% latin9 (ISO-8859-9) = latin1+"Euro sign"
\usepackage[utf8]{inputenc}	
%\PassOptionsToPackage{swedish}{babel}   % change this to your language(s)
% Spanish languages need extra options in order to work with this template
%\PassOptionsToPackage{spanish,es-lcroman}{babel}
 \usepackage[swedish]{babel}

\PassOptionsToPackage{square,numbers}{natbib}
\usepackage{natbib}
\PassOptionsToPackage{fleqn}{amsmath}		% math environments and more by the AMS 
 \usepackage{amsmath}

%\newunicodechar{00E4}{ä}
% ******************************************************************** 
% General useful packages
% ******************************************************************** 
%\PassOptionsToPackage{T1}{fontenc} % T2A for cyrillics
	\usepackage[T1]{fontenc}
\usepackage{textcomp} % fix warning with missing font shapes 
\usepackage{xspace} % to get the spacing after macros right  
\usepackage{mparhack} % get marginpar right
\usepackage{fixltx2e} % fixes some LaTeX stuff 
\PassOptionsToPackage{printonlyused,smaller}{acronym}
	\usepackage{acronym} % nice macros for handling all acronyms in the thesis
%\renewcommand*{\acsfont}[1]{\textssc{#1}} % for MinionPro
\renewcommand{\bflabel}[1]{{#1}\hfill} % fix the list of acronyms
% ****************************************************************************************************


% ****************************************************************************************************
% 4. Setup floats: tables, (sub)figures, and captions
% ****************************************************************************************************
\usepackage{tabularx} % better tables
	\setlength{\extrarowheight}{3pt} % increase table row height
\newcommand{\tableheadline}[1]{\multicolumn{1}{c}{\spacedlowsmallcaps{#1}}}
\newcommand{\myfloatalign}{\centering} % to be used with each float for alignment
\usepackage{caption}
\captionsetup{format=hang,font=small}
\usepackage{subfig}  
% ****************************************************************************************************


% ****************************************************************************************************
% 5. Setup code listings
% ****************************************************************************************************
\usepackage{listings} 
%\lstset{emph={trueIndex,root},emphstyle=\color{BlueViolet}}%\underbar} % for special keywords
\lstset{language=[LaTeX]Tex,%C++,
    keywordstyle=\color{RoyalBlue},%\bfseries,
    basicstyle=\small\ttfamily,
    %identifierstyle=\color{NavyBlue},
    commentstyle=\color{Green}\ttfamily,
    stringstyle=\rmfamily,
    numbers=none,%left,%
    numberstyle=\scriptsize,%\tiny
    stepnumber=5,
    numbersep=8pt,
    showstringspaces=false,
    breaklines=true,
    frameround=ftff,
    frame=single,
    belowcaptionskip=.75\baselineskip
    %frame=L
} 
% ****************************************************************************************************    		   


% ****************************************************************************************************
% 6. PDFLaTeX, hyperreferences and citation backreferences
% ****************************************************************************************************
% ********************************************************************
% Using PDFLaTeX
% ********************************************************************
\PassOptionsToPackage{pdftex,hyperfootnotes=false,pdfpagelabels}{hyperref}
	\usepackage{hyperref}  % backref linktocpage pagebackref
\pdfcompresslevel=9
\pdfadjustspacing=1 
\PassOptionsToPackage{pdftex}{graphicx}
	\usepackage{graphicx} 

% ********************************************************************
% Setup the style of the backrefs from the bibliography
% (translate the options to any language you use)
% ********************************************************************
\newcommand{\backrefnotcitedstring}{\relax}%(Not cited.)
\newcommand{\backrefcitedsinglestring}[1]{(Cited on page~#1.)}
\newcommand{\backrefcitedmultistring}[1]{(Cited on pages~#1.)}
\ifthenelse{\boolean{enable-backrefs}}%
{%
		\PassOptionsToPackage{hyperpageref}{backref}
		\usepackage{backref} % to be loaded after hyperref package 
		   \renewcommand{\backreftwosep}{ and~} % separate 2 pages
		   \renewcommand{\backreflastsep}{, and~} % separate last of longer list
		   \renewcommand*{\backref}[1]{}  % disable standard
		   \renewcommand*{\backrefalt}[4]{% detailed backref
		      \ifcase #1 %
		         \backrefnotcitedstring%
		      \or%
		         \backrefcitedsinglestring{#2}%
		      \else%
		         \backrefcitedmultistring{#2}%
		      \fi}%
}{\relax}    

% ********************************************************************
% Hyperreferences
% ********************************************************************
\hypersetup{%
    %draft,	% = no hyperlinking at all (useful in b/w printouts)
    colorlinks=true, linktocpage=true, pdfstartpage=3, pdfstartview=FitV,%
    % uncomment the following line if you want to have black links (e.g., for printing)
    %colorlinks=false, linktocpage=false, pdfborder={0 0 0}, pdfstartpage=3, pdfstartview=FitV,% 
    breaklinks=true, pdfpagemode=UseNone, pageanchor=true, pdfpagemode=UseOutlines,%
    plainpages=false, bookmarksnumbered, bookmarksopen=true, bookmarksopenlevel=1,%
    hypertexnames=true, pdfhighlight=/O,%nesting=true,%frenchlinks,%
    urlcolor=webbrown, linkcolor=RoyalBlue, citecolor=webgreen, %pagecolor=RoyalBlue,%
    %urlcolor=Black, linkcolor=Black, citecolor=Black, %pagecolor=Black,%
    pdftitle={\myTitle},%
    pdfauthor={\textcopyright\ \myName, \myUni, \myFaculty},%
    pdfsubject={},%
    pdfkeywords={},%
    pdfcreator={pdfLaTeX},%
    pdfproducer={LaTeX with hyperref and classicthesis}%
}   
%\hypersetup{tex4ht}


% ********************************************************************
% Setup autoreferences
% ********************************************************************
% There are some issues regarding autorefnames
% http://www.ureader.de/msg/136221647.aspx
% http://www.tex.ac.uk/cgi-bin/texfaq2html?label=latexwords
% you have to redefine the makros for the 
% language you use, e.g., american, ngerman
% (as chosen when loading babel/AtBeginDocument)
% ********************************************************************
\makeatletter
\@ifpackageloaded{babel}%
    {%
       \addto\extrasamerican{%
					\renewcommand*{\figureautorefname}{Figure}%
					\renewcommand*{\tableautorefname}{Table}%
					\renewcommand*{\partautorefname}{Part}%
					\renewcommand*{\chapterautorefname}{Chapter}%
					\renewcommand*{\sectionautorefname}{Section}%
					\renewcommand*{\subsectionautorefname}{Section}%
					\renewcommand*{\subsubsectionautorefname}{Section}% 	
				}%
       \addto\extrasngerman{% 
					\renewcommand*{\paragraphautorefname}{Absatz}%
					\renewcommand*{\subparagraphautorefname}{Unterabsatz}%
					\renewcommand*{\footnoteautorefname}{Fu\"snote}%
					\renewcommand*{\FancyVerbLineautorefname}{Zeile}%
					\renewcommand*{\theoremautorefname}{Theorem}%
					\renewcommand*{\appendixautorefname}{Anhang}%
					\renewcommand*{\equationautorefname}{Gleichung}%        
					\renewcommand*{\itemautorefname}{Punkt}%
				}%	
			% Fix to getting autorefs for subfigures right (thanks to Belinda Vogt for changing the definition)
			\providecommand{\subfigureautorefname}{\figureautorefname}%  			
    }{\relax}
\makeatother


% ****************************************************************************************************
% 7. Last calls before the bar closes
% ****************************************************************************************************
% ********************************************************************
% Development Stuff
% ********************************************************************
\listfiles
%\PassOptionsToPackage{l2tabu,orthodox,abort}{nag}
%	\usepackage{nag}
%\PassOptionsToPackage{warning, all}{onlyamsmath}
%	\usepackage{onlyamsmath}

% ********************************************************************
% Last, but not least...
% ********************************************************************
\usepackage{classicthesis} 
% ****************************************************************************************************


% ****************************************************************************************************
% 8. Further adjustments (experimental)
% ****************************************************************************************************
% ********************************************************************
% Changing the text area
% ********************************************************************
%\linespread{1.05} % a bit more for Palatino
%\areaset[current]{312pt}{761pt} % 686 (factor 2.2) + 33 head + 42 head \the\footskip
%\setlength{\marginparwidth}{7em}%
%\setlength{\marginparsep}{2em}%

% ********************************************************************
% Using different fonts
% ********************************************************************
%\usepackage[oldstylenums]{kpfonts} % oldstyle notextcomp
%\usepackage[osf]{libertine}
%\usepackage{hfoldsty} % Computer Modern with osf
%\usepackage[light,condensed,math]{iwona}
%\renewcommand{\sfdefault}{iwona}
%\usepackage{lmodern} % <-- no osf support :-(
%\usepackage[urw-garamond]{mathdesign} <-- no osf support :-(
% ****************************************************************************************************

% *********************************************************************
% Syntax highlighting
% *********************************************************************
% \usepackage{fontspec}
\usepackage{minted}

%
% Fix to support extended characters in minted with combination of pdfTeX
% http://tex.stackexchange.com/questions/19781/using-package-minted-with-non-english-characters
%

\makeatletter
\newcommand{\minted@write@detok}[1]{%
  \immediate\write\FV@OutFile{\detokenize{#1}}}%

% Blank page
\newcommand*{\blankpage}{%
  \begingroup
    \setlength{\topskip}{0pt}%
    \setlength{\parskip}{0pt}%
    \vspace*{\fill}%
    \centering
    \nointerlineskip % suppresses partial \baselineskip above text
    This page is intentionally left (almost) blank.\par
    \thispagestyle{empty}
    \setcounter{page}{}
    \vspace{\fill}%
  \endgroup
  % \showlists
}

\newcommand{\minted@FVB@VerbatimOut}[1]{%
  \@bsphack
  \begingroup
    \FV@UseKeyValues
    \FV@DefineWhiteSpace
    \def\FV@Space{\space}%
    \FV@DefineTabOut
    %\def\FV@ProcessLine{\immediate\write\FV@OutFile}% %Old, non-Unicode version
    \let\FV@ProcessLine\minted@write@detok %Patch for Unicode
    \immediate\openout\FV@OutFile #1\relax
    \let\FV@FontScanPrep\relax
%% DG/SR modification begin - May. 18, 1998 (to avoid problems with ligatures)
    \let\@noligs\relax
%% DG/SR modification end
    \FV@Scan}
    \let\FVB@VerbatimOut\minted@FVB@VerbatimOut

\renewcommand\minted@savecode[1]{
  \immediate\openout\minted@code\jobname.pyg
  \immediate\write\minted@code{\expandafter\detokenize\expandafter{#1}}%
  \immediate\closeout\minted@code}
\makeatother

\usepackage{tcolorbox}
\tcbuselibrary{skins}
\usetikzlibrary{shadings}

\usepackage{etoolbox}

% Code
\BeforeBeginEnvironment{code}{\begin{tcolorbox}[enhanced,
width=\linewidth,
enlarge top by=3pt,enlarge bottom by=3pt,
enlarge left by=3pt,enlarge right by=3pt,
frame hidden,boxrule=0pt,top=1mm,bottom=1mm,
colframe=green!30!black, colbacktitle=green!50!yellow,
coltitle=black, colback=bg,
borderline={.1pt}{-0.5pt}{gray, dashed, sharp corners}]}%
\AfterEndEnvironment{code}{\end{tcolorbox}}

% Terminal
\BeforeBeginEnvironment{terminal}{\begin{tcolorbox}[enhanced,
width=\linewidth,
enlarge top by=3pt,enlarge bottom by=3pt,
enlarge left by=3pt,enlarge right by=3pt,
frame hidden,boxrule=0pt,top=1mm,bottom=1mm,
colframe=green!30!black, colbacktitle=green!50!yellow,
coltitle=black, colback=bg,
borderline={.1pt}{-0.5pt}{gray, dashed, sharp corners}]}%
\AfterEndEnvironment{terminal}{\end{tcolorbox}}

% Java
\BeforeBeginEnvironment{java-code}{\begin{tcolorbox}[enhanced,
width=\linewidth,
enlarge top by=3pt,enlarge bottom by=3pt,
enlarge left by=3pt,enlarge right by=3pt,
frame hidden,boxrule=0pt,top=1mm,bottom=1mm,
colframe=green!30!black, colbacktitle=green!50!yellow,
coltitle=black, colback=bg,
borderline={.1pt}{-0.5pt}{gray, dashed, sharp corners}]}%
\AfterEndEnvironment{java-code}{\end{tcolorbox}}

% Js
\BeforeBeginEnvironment{js-code}{\begin{tcolorbox}[enhanced,
width=\linewidth,
enlarge top by=3pt,enlarge bottom by=3pt,
enlarge left by=3pt,enlarge right by=3pt,
frame hidden,boxrule=0pt,top=1mm,bottom=1mm,
colframe=green!30!black, colbacktitle=green!50!yellow,
coltitle=black, colback=bg,
borderline={.1pt}{-0.5pt}{gray, dashed, sharp corners}]}%
\AfterEndEnvironment{js-code}{\end{tcolorbox}}

% Json
\BeforeBeginEnvironment{json-normal}{\begin{tcolorbox}[enhanced,
width=\linewidth,
enlarge top by=3pt,enlarge bottom by=3pt,
enlarge left by=3pt,enlarge right by=3pt,
frame hidden,boxrule=0pt,top=1mm,bottom=1mm,
colframe=green!30!black, colbacktitle=green!50!yellow,
coltitle=black, colback=bg,
borderline={.1pt}{-0.5pt}{gray, dashed, sharp corners}]}%
\AfterEndEnvironment{json-normal}{\end{tcolorbox}}

% Json
\BeforeBeginEnvironment{json-small}{\begin{tcolorbox}[enhanced,
width=\linewidth,
enlarge top by=3pt,enlarge bottom by=3pt,
enlarge left by=3pt,enlarge right by=3pt,
frame hidden,boxrule=0pt,top=1mm,bottom=1mm,
colframe=green!30!black, colbacktitle=green!50!yellow,
coltitle=black, colback=bg,
borderline={.1pt}{-0.5pt}{gray, dashed, sharp corners}]}%
\AfterEndEnvironment{json-small}{\end{tcolorbox}}


% Plaintext
\BeforeBeginEnvironment{plaintext}{\begin{tcolorbox}[enhanced,
frame hidden,boxrule=0pt,top=1mm,bottom=0.5mm]}%
\AfterEndEnvironment{plaintext}{\end{tcolorbox}}%

\definecolor{bg}{HTML}{F2F2F2}
\newminted[code]{text}{mathescape=true,linenos=true,numbersep=5pt,fontsize=\footnotesize,framesep=10pt}
\newminted[java-code]{csharp}{linenos=true,numbersep=5pt,fontsize=\footnotesize,framesep=10pt}
\newminted[js-code]{js}{linenos=true,numbersep=5pt,fontsize=\footnotesize,framesep=10pt}
\newminted[json-normal]{js}{linenos=true,numbersep=5pt,fontsize=\footnotesize,framesep=10pt}
\newminted[json-small]{js}{linenos=true,numbersep=5pt,fontsize=\scriptsize,framesep=10pt}
\newminted[terminal]{text}{linenos=false,numbersep=5pt,fontsize=\footnotesize,framesep=10pt}
\newminted[plaintext]{text}{linenos=false,numbersep=5pt,fontsize=\footnotesize,framesep=10pt}
\newminted[bashcode2]{bash}{linenos=true, texcl=true}
\usepackage{scrhack} % fix warnings when using KOMA with listings package         
%\setsansfont{Calibri}
%\setmonofont{Consolas}
%\renewcommand{\familydefault}{\sfdefault}

\usepackage{etoolbox}

% Margin
\usepackage[left=1.5in,right=1.5in,top=1.5in,bottom=1.5in]{geometry}
\usepackage[framemethod=default]{mdframed}

%Exact positioning
\usepackage{float}

%Hyper-links
\usepackage{hyperref}

%Todo color
\usepackage{xcolor}
\newcommand\todo[1]{\textcolor{red}{#1}}

% Centering of graphics
\usepackage[export]{adjustbox}% http://ctan.org/pkg/adjustbox

%Hidden sections
\newcommand{\hiddensubsection}[1]{
    \stepcounter{subsection}
    \subsection*{\arabic{chapter}.\arabic{section}.\arabic{subsection}\hspace{1em}{#1}}
}

\newcommand{\hiddensubsubsection}[1]{
    \stepcounter{subsubsection}
    \subsubsection*{\arabic{chapter}.\arabic{section}.\arabic{subsection}\hspace{1em}{#1}}
}

\newcommand{\hiddensubsubsubsection}[1]{
    \stepcounter{subsubsubsection}
    \subsubsubsection*{\arabic{chapter}.\arabic{section}.\arabic{subsection}\hspace{1em}{#1}}
}

% Include pdfs
\usepackage{pdfpages}
% ****************************************************************************************************  
% If you like the classicthesis, then I would appreciate a postcard. 
% My address can be found in the file ClassicThesis.pdf. A collection 
% of the postcards I received so far is available online at 
% http://postcards.miede.de
% ****************************************************************************************************

% ****************************************************************************************************
% 1. Configure classicthesis for your needs here, e.g., remove "drafting" below 
% in order to deactivate the time-stamp on the pages
% ****************************************************************************************************
\PassOptionsToPackage{eulerchapternumbers,listings,drafting,%
				 pdfspacing,%floatperchapter,%linedheaders,%
				 subfig,beramono,eulermath,parts}{classicthesis}										
% ********************************************************************
% Available options for classicthesis.sty 
% (see ClassicThesis.pdf for more information):
% drafting
% parts nochapters linedheaders
% eulerchapternumbers beramono eulermath pdfspacing minionprospacing
% tocaligned dottedtoc manychapters
% listings floatperchapter subfig
% ********************************************************************

% ********************************************************************
% Triggers for this config
% ******************************************************************** 
\usepackage{ifthen}
\newboolean{enable-backrefs} % enable backrefs in the bibliography
\setboolean{enable-backrefs}{false} % true false
% ****************************************************************************************************


% ****************************************************************************************************
% 2. Personal data and user ad-hoc commands
% ****************************************************************************************************
\newcommand{\myTitle}{A Classic Thesis Style\xspace}
\newcommand{\mySubtitle}{An Homage to The Elements of Typographic Style\xspace}
\newcommand{\myDegree}{Doktor-Ingenieur (Dr.-Ing.)\xspace}
\newcommand{\myName}{Andr\'e Miede\xspace}
\newcommand{\myProf}{Put name here\xspace}
\newcommand{\myOtherProf}{Put name here\xspace}
\newcommand{\mySupervisor}{Put name here\xspace}
\newcommand{\myFaculty}{Put data here\xspace}
\newcommand{\myDepartment}{Put data here\xspace}
\newcommand{\myUni}{Put data here\xspace}
\newcommand{\myLocation}{Darmstadt\xspace}
\newcommand{\myTime}{August 2012\xspace}
\newcommand{\myVersion}{version 4.1\xspace}

% ********************************************************************
% Setup, finetuning, and useful commands
% ********************************************************************
\newcounter{dummy} % necessary for correct hyperlinks (to index, bib, etc.)
\newlength{\abcd} % for ab..z string length calculation
\providecommand{\mLyX}{L\kern-.1667em\lower.25em\hbox{Y}\kern-.125emX\@}
\newcommand{\ie}{i.\,e.}
\newcommand{\Ie}{I.\,e.}
\newcommand{\eg}{e.\,g.}
\newcommand{\Eg}{E.\,g.} 
% ****************************************************************************************************


% ****************************************************************************************************
% 3. Loading some handy packages
% ****************************************************************************************************
% ******************************************************************** 
% Packages with options that might require adjustments
% ******************************************************************** 
%\PassOptionsToPackage{utf8}{inputenc}	% latin9 (ISO-8859-9) = latin1+"Euro sign"
\usepackage[utf8]{inputenc}	
%\PassOptionsToPackage{swedish}{babel}   % change this to your language(s)
% Spanish languages need extra options in order to work with this template
%\PassOptionsToPackage{spanish,es-lcroman}{babel}
 \usepackage[swedish]{babel}

\PassOptionsToPackage{square,numbers}{natbib}
\usepackage{natbib}
\PassOptionsToPackage{fleqn}{amsmath}		% math environments and more by the AMS 
 \usepackage{amsmath}

%\newunicodechar{00E4}{ä}
% ******************************************************************** 
% General useful packages
% ******************************************************************** 
%\PassOptionsToPackage{T1}{fontenc} % T2A for cyrillics
	\usepackage[T1]{fontenc}
\usepackage{textcomp} % fix warning with missing font shapes 
\usepackage{xspace} % to get the spacing after macros right  
\usepackage{mparhack} % get marginpar right
\usepackage{fixltx2e} % fixes some LaTeX stuff 
\PassOptionsToPackage{printonlyused,smaller}{acronym}
	\usepackage{acronym} % nice macros for handling all acronyms in the thesis
%\renewcommand*{\acsfont}[1]{\textssc{#1}} % for MinionPro
\renewcommand{\bflabel}[1]{{#1}\hfill} % fix the list of acronyms
% ****************************************************************************************************


% ****************************************************************************************************
% 4. Setup floats: tables, (sub)figures, and captions
% ****************************************************************************************************
\usepackage{tabularx} % better tables
	\setlength{\extrarowheight}{3pt} % increase table row height
\newcommand{\tableheadline}[1]{\multicolumn{1}{c}{\spacedlowsmallcaps{#1}}}
\newcommand{\myfloatalign}{\centering} % to be used with each float for alignment
\usepackage{caption}
\captionsetup{format=hang,font=small}
\usepackage{subfig}  
% ****************************************************************************************************


% ****************************************************************************************************
% 5. Setup code listings
% ****************************************************************************************************
\usepackage{listings} 
%\lstset{emph={trueIndex,root},emphstyle=\color{BlueViolet}}%\underbar} % for special keywords
\lstset{language=[LaTeX]Tex,%C++,
    keywordstyle=\color{RoyalBlue},%\bfseries,
    basicstyle=\small\ttfamily,
    %identifierstyle=\color{NavyBlue},
    commentstyle=\color{Green}\ttfamily,
    stringstyle=\rmfamily,
    numbers=none,%left,%
    numberstyle=\scriptsize,%\tiny
    stepnumber=5,
    numbersep=8pt,
    showstringspaces=false,
    breaklines=true,
    frameround=ftff,
    frame=single,
    belowcaptionskip=.75\baselineskip
    %frame=L
} 
% ****************************************************************************************************    		   


% ****************************************************************************************************
% 6. PDFLaTeX, hyperreferences and citation backreferences
% ****************************************************************************************************
% ********************************************************************
% Using PDFLaTeX
% ********************************************************************
\PassOptionsToPackage{pdftex,hyperfootnotes=false,pdfpagelabels}{hyperref}
	\usepackage{hyperref}  % backref linktocpage pagebackref
\pdfcompresslevel=9
\pdfadjustspacing=1 
\PassOptionsToPackage{pdftex}{graphicx}
	\usepackage{graphicx} 

% ********************************************************************
% Setup the style of the backrefs from the bibliography
% (translate the options to any language you use)
% ********************************************************************
\newcommand{\backrefnotcitedstring}{\relax}%(Not cited.)
\newcommand{\backrefcitedsinglestring}[1]{(Cited on page~#1.)}
\newcommand{\backrefcitedmultistring}[1]{(Cited on pages~#1.)}
\ifthenelse{\boolean{enable-backrefs}}%
{%
		\PassOptionsToPackage{hyperpageref}{backref}
		\usepackage{backref} % to be loaded after hyperref package 
		   \renewcommand{\backreftwosep}{ and~} % separate 2 pages
		   \renewcommand{\backreflastsep}{, and~} % separate last of longer list
		   \renewcommand*{\backref}[1]{}  % disable standard
		   \renewcommand*{\backrefalt}[4]{% detailed backref
		      \ifcase #1 %
		         \backrefnotcitedstring%
		      \or%
		         \backrefcitedsinglestring{#2}%
		      \else%
		         \backrefcitedmultistring{#2}%
		      \fi}%
}{\relax}    

% ********************************************************************
% Hyperreferences
% ********************************************************************
\hypersetup{%
    %draft,	% = no hyperlinking at all (useful in b/w printouts)
    colorlinks=true, linktocpage=true, pdfstartpage=3, pdfstartview=FitV,%
    % uncomment the following line if you want to have black links (e.g., for printing)
    %colorlinks=false, linktocpage=false, pdfborder={0 0 0}, pdfstartpage=3, pdfstartview=FitV,% 
    breaklinks=true, pdfpagemode=UseNone, pageanchor=true, pdfpagemode=UseOutlines,%
    plainpages=false, bookmarksnumbered, bookmarksopen=true, bookmarksopenlevel=1,%
    hypertexnames=true, pdfhighlight=/O,%nesting=true,%frenchlinks,%
    urlcolor=webbrown, linkcolor=RoyalBlue, citecolor=webgreen, %pagecolor=RoyalBlue,%
    %urlcolor=Black, linkcolor=Black, citecolor=Black, %pagecolor=Black,%
    pdftitle={\myTitle},%
    pdfauthor={\textcopyright\ \myName, \myUni, \myFaculty},%
    pdfsubject={},%
    pdfkeywords={},%
    pdfcreator={pdfLaTeX},%
    pdfproducer={LaTeX with hyperref and classicthesis}%
}   
%\hypersetup{tex4ht}


% ********************************************************************
% Setup autoreferences
% ********************************************************************
% There are some issues regarding autorefnames
% http://www.ureader.de/msg/136221647.aspx
% http://www.tex.ac.uk/cgi-bin/texfaq2html?label=latexwords
% you have to redefine the makros for the 
% language you use, e.g., american, ngerman
% (as chosen when loading babel/AtBeginDocument)
% ********************************************************************
\makeatletter
\@ifpackageloaded{babel}%
    {%
       \addto\extrasamerican{%
					\renewcommand*{\figureautorefname}{Figure}%
					\renewcommand*{\tableautorefname}{Table}%
					\renewcommand*{\partautorefname}{Part}%
					\renewcommand*{\chapterautorefname}{Chapter}%
					\renewcommand*{\sectionautorefname}{Section}%
					\renewcommand*{\subsectionautorefname}{Section}%
					\renewcommand*{\subsubsectionautorefname}{Section}% 	
				}%
       \addto\extrasngerman{% 
					\renewcommand*{\paragraphautorefname}{Absatz}%
					\renewcommand*{\subparagraphautorefname}{Unterabsatz}%
					\renewcommand*{\footnoteautorefname}{Fu\"snote}%
					\renewcommand*{\FancyVerbLineautorefname}{Zeile}%
					\renewcommand*{\theoremautorefname}{Theorem}%
					\renewcommand*{\appendixautorefname}{Anhang}%
					\renewcommand*{\equationautorefname}{Gleichung}%        
					\renewcommand*{\itemautorefname}{Punkt}%
				}%	
			% Fix to getting autorefs for subfigures right (thanks to Belinda Vogt for changing the definition)
			\providecommand{\subfigureautorefname}{\figureautorefname}%  			
    }{\relax}
\makeatother


% ****************************************************************************************************
% 7. Last calls before the bar closes
% ****************************************************************************************************
% ********************************************************************
% Development Stuff
% ********************************************************************
\listfiles
%\PassOptionsToPackage{l2tabu,orthodox,abort}{nag}
%	\usepackage{nag}
%\PassOptionsToPackage{warning, all}{onlyamsmath}
%	\usepackage{onlyamsmath}

% ********************************************************************
% Last, but not least...
% ********************************************************************
\usepackage{classicthesis} 
% ****************************************************************************************************


% ****************************************************************************************************
% 8. Further adjustments (experimental)
% ****************************************************************************************************
% ********************************************************************
% Changing the text area
% ********************************************************************
%\linespread{1.05} % a bit more for Palatino
%\areaset[current]{312pt}{761pt} % 686 (factor 2.2) + 33 head + 42 head \the\footskip
%\setlength{\marginparwidth}{7em}%
%\setlength{\marginparsep}{2em}%

% ********************************************************************
% Using different fonts
% ********************************************************************
%\usepackage[oldstylenums]{kpfonts} % oldstyle notextcomp
%\usepackage[osf]{libertine}
%\usepackage{hfoldsty} % Computer Modern with osf
%\usepackage[light,condensed,math]{iwona}
%\renewcommand{\sfdefault}{iwona}
%\usepackage{lmodern} % <-- no osf support :-(
%\usepackage[urw-garamond]{mathdesign} <-- no osf support :-(
% ****************************************************************************************************

% *********************************************************************
% Syntax highlighting
% *********************************************************************
% \usepackage{fontspec}
\usepackage{minted}

%
% Fix to support extended characters in minted with combination of pdfTeX
% http://tex.stackexchange.com/questions/19781/using-package-minted-with-non-english-characters
%

\makeatletter
\newcommand{\minted@write@detok}[1]{%
  \immediate\write\FV@OutFile{\detokenize{#1}}}%

% Blank page
\newcommand*{\blankpage}{%
  \begingroup
    \setlength{\topskip}{0pt}%
    \setlength{\parskip}{0pt}%
    \vspace*{\fill}%
    \centering
    \nointerlineskip % suppresses partial \baselineskip above text
    This page is intentionally left (almost) blank.\par
    \vspace{\fill}%
  \endgroup
  % \showlists
}

\newcommand{\minted@FVB@VerbatimOut}[1]{%
  \@bsphack
  \begingroup
    \FV@UseKeyValues
    \FV@DefineWhiteSpace
    \def\FV@Space{\space}%
    \FV@DefineTabOut
    %\def\FV@ProcessLine{\immediate\write\FV@OutFile}% %Old, non-Unicode version
    \let\FV@ProcessLine\minted@write@detok %Patch for Unicode
    \immediate\openout\FV@OutFile #1\relax
    \let\FV@FontScanPrep\relax
%% DG/SR modification begin - May. 18, 1998 (to avoid problems with ligatures)
    \let\@noligs\relax
%% DG/SR modification end
    \FV@Scan}
    \let\FVB@VerbatimOut\minted@FVB@VerbatimOut

\renewcommand\minted@savecode[1]{
  \immediate\openout\minted@code\jobname.pyg
  \immediate\write\minted@code{\expandafter\detokenize\expandafter{#1}}%
  \immediate\closeout\minted@code}
\makeatother

\usepackage{tcolorbox}
\tcbuselibrary{skins}
\usetikzlibrary{shadings}

\usepackage{etoolbox}

% Code
\BeforeBeginEnvironment{code}{\begin{tcolorbox}[enhanced,
width=\linewidth,
enlarge top by=3pt,enlarge bottom by=3pt,
enlarge left by=3pt,enlarge right by=3pt,
frame hidden,boxrule=0pt,top=1mm,bottom=1mm,
colframe=green!30!black, colbacktitle=green!50!yellow,
coltitle=black, colback=bg,
borderline={.1pt}{-0.5pt}{gray, dashed, sharp corners}]}%
\AfterEndEnvironment{code}{\end{tcolorbox}}

% Terminal
\BeforeBeginEnvironment{terminal}{\begin{tcolorbox}[enhanced,
width=\linewidth,
enlarge top by=3pt,enlarge bottom by=3pt,
enlarge left by=3pt,enlarge right by=3pt,
frame hidden,boxrule=0pt,top=1mm,bottom=1mm,
colframe=green!30!black, colbacktitle=green!50!yellow,
coltitle=black, colback=bg,
borderline={.1pt}{-0.5pt}{gray, dashed, sharp corners}]}%
\AfterEndEnvironment{terminal}{\end{tcolorbox}}

% Java
\BeforeBeginEnvironment{java-code}{\begin{tcolorbox}[enhanced,
width=\linewidth,
enlarge top by=3pt,enlarge bottom by=3pt,
enlarge left by=3pt,enlarge right by=3pt,
frame hidden,boxrule=0pt,top=1mm,bottom=1mm,
colframe=green!30!black, colbacktitle=green!50!yellow,
coltitle=black, colback=bg,
borderline={.1pt}{-0.5pt}{gray, dashed, sharp corners}]}%
\AfterEndEnvironment{java-code}{\end{tcolorbox}}

% Js
\BeforeBeginEnvironment{js-code}{\begin{tcolorbox}[enhanced,
width=\linewidth,
enlarge top by=3pt,enlarge bottom by=3pt,
enlarge left by=3pt,enlarge right by=3pt,
frame hidden,boxrule=0pt,top=1mm,bottom=1mm,
colframe=green!30!black, colbacktitle=green!50!yellow,
coltitle=black, colback=bg,
borderline={.1pt}{-0.5pt}{gray, dashed, sharp corners}]}%
\AfterEndEnvironment{js-code}{\end{tcolorbox}}

% Json
\BeforeBeginEnvironment{json-normal}{\begin{tcolorbox}[enhanced,
width=\linewidth,
enlarge top by=3pt,enlarge bottom by=3pt,
enlarge left by=3pt,enlarge right by=3pt,
frame hidden,boxrule=0pt,top=1mm,bottom=1mm,
colframe=green!30!black, colbacktitle=green!50!yellow,
coltitle=black, colback=bg,
borderline={.1pt}{-0.5pt}{gray, dashed, sharp corners}]}%
\AfterEndEnvironment{json-normal}{\end{tcolorbox}}

% Json
\BeforeBeginEnvironment{json-small}{\begin{tcolorbox}[enhanced,
width=\linewidth,
enlarge top by=3pt,enlarge bottom by=3pt,
enlarge left by=3pt,enlarge right by=3pt,
frame hidden,boxrule=0pt,top=1mm,bottom=1mm,
colframe=green!30!black, colbacktitle=green!50!yellow,
coltitle=black, colback=bg,
borderline={.1pt}{-0.5pt}{gray, dashed, sharp corners}]}%
\AfterEndEnvironment{json-small}{\end{tcolorbox}}


% Plaintext
\BeforeBeginEnvironment{plaintext}{\begin{tcolorbox}[enhanced,
frame hidden,boxrule=0pt,top=1mm,bottom=0.5mm]}%
\AfterEndEnvironment{plaintext}{\end{tcolorbox}}%

\definecolor{bg}{HTML}{F2F2F2}
\newminted[code]{text}{mathescape=true,linenos=true,numbersep=5pt,fontsize=\footnotesize,framesep=10pt}
\newminted[java-code]{csharp}{linenos=true,numbersep=5pt,fontsize=\footnotesize,framesep=10pt}
\newminted[js-code]{js}{linenos=true,numbersep=5pt,fontsize=\footnotesize,framesep=10pt}
\newminted[json-normal]{js}{linenos=true,numbersep=5pt,fontsize=\footnotesize,framesep=10pt}
\newminted[json-small]{js}{linenos=true,numbersep=5pt,fontsize=\scriptsize,framesep=10pt}
\newminted[terminal]{text}{linenos=false,numbersep=5pt,fontsize=\footnotesize,framesep=10pt}
\newminted[plaintext]{text}{linenos=false,numbersep=5pt,fontsize=\footnotesize,framesep=10pt}
\newminted[bashcode2]{bash}{linenos=true, texcl=true}
\usepackage{scrhack} % fix warnings when using KOMA with listings package         
%\setsansfont{Calibri}
%\setmonofont{Consolas}
%\renewcommand{\familydefault}{\sfdefault}

\usepackage{etoolbox}

% Margin
\usepackage[left=1.5in,right=1.5in,top=1.5in,bottom=1.5in]{geometry}
\usepackage[framemethod=default]{mdframed}

%Exact positioning
\usepackage{float}

%Hyper-links
\usepackage{hyperref}

%Todo color
\usepackage{xcolor}
\newcommand\todo[1]{\textcolor{red}{#1}}

% Centering of graphics
\usepackage[export]{adjustbox}% http://ctan.org/pkg/adjustbox

%Hidden sections
\newcommand{\hiddensubsection}[1]{
    \stepcounter{subsection}
    \subsection*{\arabic{chapter}.\arabic{section}.\arabic{subsection}\hspace{1em}{#1}}
}

\newcommand{\hiddensubsubsection}[1]{
    \stepcounter{subsubsection}
    \subsubsection*{\arabic{chapter}.\arabic{section}.\arabic{subsection}\hspace{1em}{#1}}
}

\newcommand{\hiddensubsubsubsection}[1]{
    \stepcounter{subsubsubsection}
    \subsubsubsection*{\arabic{chapter}.\arabic{section}.\arabic{subsection}\hspace{1em}{#1}}
}