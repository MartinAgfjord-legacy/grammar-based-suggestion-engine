%************************************************
\chapter{Application development}\label{ch:application-overview}
%************************************************

\section{Requirements specification}

\subsection{Generation of mock data}
As described in \autoref{sec:problem-description}, we want to develop an application which can translate natural language questions that refers to entities in a database or index owned by a software development company. This project has been made with strong collaboration with Findwise, a company with focus on search driven solutions. Although Findwise has an index with information about employees, projects and customers, their information is confidential and cannot be published in a master thesis. A different approach to get hold of relevant data is to generate mock data that is inspired by Findwise's data model.

\subsection{Grammar development}
The grammar in section 1 can only translate the question \texttt{people who know Java} in English and Swedish into Solr query language. The grammar needs to be extended to handle \emph{any} programming language that exists in the mock data, not only \texttt{Java}. In addition, it also needs to support other questions involving not only people, but customers and projects.

\subsection{Suggestions}
If a user have no idea of which questions the application can translate, how can she use the application? GF can only parse syntactically correct input, therefore, if the question has one character in the wrong place it will not be able to translate anything.

A suggestion engine can be used to complete the input of the user. For instance, if a user types \texttt{'All persons that know Java'} the application can display a list of related valid sentences where the user can choose which questions that she wants to replace the original sentence with.

To make the application more flexible, a better suggestion engine shall be able to complete sentences based purely on keywords, for example \texttt{'people java'} shall suggest sentences that can be formulated with these two words.

\subsection{Runtime environment}
The chosen programming language for this project is Java. The main reason is because it is Findwise's primary programming language. It is also very well known among many companies in the world.

Many professional Java-developers adopts a specific development platform, \ac{javaee}. This platform provides many libraries that can be scaled to work in an enterprise environment. This project also adopts Java EE.

A typical Java EE project make use of several libraries, in computer science terms we say that a project can have other libraries as \emph{dependencies}. It is not unusual that these libraries also have their own dependencies. Larger projects can therefore have a lot of dependencies, so many that it becomes hard to keep track of them. This project make use of a tool called Maven to handle dependencies. One simply list all libraries the project shall have access to, then Maven will automatically fetch them and their dependencies. This also makes the application more flexible, as it do not have to include the needed libraries in the application.

\todo{Some text about tomcat}

Details about the runtime environment can be found in \autoref{ch:appendix-a}.

\section{Generation of mock data}

\section{Grammar development}

\section{Suggestion engine}