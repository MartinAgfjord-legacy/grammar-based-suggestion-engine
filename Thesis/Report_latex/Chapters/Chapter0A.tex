%********************************************************************
% Appendix
%*******************************************************
% If problems with the headers: get headings in appendix etc. right
%\markboth{\spacedlowsmallcaps{Appendix}}{\spacedlowsmallcaps{Appendix}}
\chapter{GF runtime systems and libraries}\label{ch:appendix-a}

\section{GF runtime systems}

While the GF-shell is a powerful tool, it is not very convenient to interact with when programming an application. It requires the hosting computer to have installed and also requires the application to have access to executing other applications, a major drawback. Luckily, the creators of GF has thought about this and built embeddable runtime systems for a few programming languages \cite[p. 3]{angelov:2011}\marginpar{The GF-shell uses the Haskell runtime system.}. These runtime systems makes it possible to interact with a grammar directly through language specific data types. We have chosen to work with the Java-runtime system in this project. The main purpose is because it is used by many developers and companies in the industry.

\subsection{Portable Grammar Format}
The GF-shell interacts with grammars by interpreting the GF programming language. This allows us to write our grammars in an simple and convenient syntax. Interpreting the GF programming language directly is however a heavy operation\cite{angelov:2011}[p. 13], especially with larger grammars. This is where the \ac{pgf}\cite{angelov:2011}[p. 14] comes in. PGF is a custom made machine language which is dynamically created by compiling a grammar with GF into a PGF-file. The runtime systems works exclusively with PGF-files.

\subsection{GF libraries}
In order to use the Java-runtime, we first need to generate a few libraries which is used by the runtime system. The Java-runtime system depends on the C-runtime system and a special wrapper between the C- and the Java-runtime. The libraries are platform dependent and at the time of writing, no pre-generated libraries exists. We therefore need to generate the libraries by ourselves. We will start by generating and installing the C-libraries. We will then go through how to generate the wrapper library.

\subsubsection{Building and installing the C-runtime}
Start by fetching the needed dependencies

\begin{terminal}
sudo apt-get install gcc autoconf libtool
\end{terminal}

Download the latest source code of GF from GitHub.

\begin{terminal}
git clone https://github.com/GrammaticalFramework/GF.git
\end{terminal}

It is also possible to download the project as an archive by visiting the repository url.

You will receive a directory \texttt{GF/}. Change the current working directory to the C-runtime folder.

\begin{terminal}
cd GF/src/runtime/c
\end{terminal}

Generate a configuration file

\begin{terminal}
autoreconf -i
\end{terminal}

Check that all dependencies are met

\begin{terminal}
./configure
\end{terminal}

If there exists a dependency that is not fulfilled, try to install an appropriate package using your package-manager.

Build the program

\begin{terminal}
make
\end{terminal}

Install the libraries you just built

\begin{terminal}
sudo make install
\end{terminal}

The C-runtime for PGF is now installed.

\subsection{Building and installing the C to Java wrapper library}
Start by installing the needed dependency

\begin{terminal}
sudo apt-get install g++
\end{terminal}

The wrapper is built by using a script which is executed in Eclipse. This step assumes that you have Eclipse installed with the CDT-plugin. If you don't have Eclipse, you can download it with your package manager, just do not forget to install the CDT-plugin.

Start Eclipse and choose \texttt{File} > \texttt{Import..} in the menu. Choose \texttt{Import Existing Projects into Workspace} and click on the \texttt{Next} button. Select \texttt{Browse...} and navigate to the location where you downloaded GF from GitHub and press enter. Uncheck everything except \texttt{jpgf} and click on \texttt{Finish}. You have now imported the project which can build the Java-runtime system. 

Unfortunately, the build-configuration for the jpgf-project is not complete at time of writing. We therefore need to make additional adjustments in order to build the project.

Right-click on the project and choose \texttt{Properties}. Click on \texttt{Includes} which is located below \texttt{GCC C Compiler}. You will see one directory listed in the textbox. You need to check that this directory exists. If not, change it to the correct one. For instance, this tutorial was written using Ubuntu 14.04 amd64 with OpenJDK 7, hence the correct directory is

\begin{terminal}
/usr/lib/jvm/java-7-openjdk-amd64/include
\end{terminal}

The project also needs another flag in order to build properly. In the \texttt{Properties}-window, click on \texttt{Miscellaneous} below \texttt{GCC C Compiler}. Add \texttt{-fPIC} to the text field next to \texttt{Other flags}. Click on \texttt{Ok} to save the settings.

You can now build the project by choosing \texttt{Project} > \texttt{Build Project} in the menu. If everything went well you shall have generated a file \texttt{libjpgf.so} in \texttt{Release (posix)/} . You can check that the dependencies of \texttt{libjpgf.so} is fulfilled (i.e. it finds the C-runtime) by executing the following in a terminal

\begin{terminal}
ldd libjpgf.so
\end{terminal}

If you cannot see \texttt{not found} anywhere in the results, all dependencies are met. However if some dependencies are missing, try to locate the files and move them to \texttt{/usr/local/lib} (or \texttt{/usr/lib} in some distros).

The last step is to move \texttt{libjpgf.so} into the correct directory.

\begin{terminal}
mv libjpgf.so /usr/local/lib (or /usr/lib)
\end{terminal}

You have now installed the wrapper library.

\subsection{Using the Java-runtime}
\todo{Don't forget how to set java lib path when using tomcat!}