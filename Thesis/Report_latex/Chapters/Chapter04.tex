%************************************************
\chapter{Conclusion}\label{ch:conclusion}
%************************************************
\section{Implications for future research}
\subsection{Inspiration to GF-developers how to handle suggestions}
A hint to GF-developers that it is crucial for an application to handle incorrect sentences properly.
\subsection{Not use RGL when focusing on one natural language}
When only using one natural language and have relatively little valid sentences, I feel that RGL is unnessecary. It makes it harder to express valid sentences, but also express invalid sentences.

\section{Known issues}
\subsection{Multiple equal abstract syntax trees and linearizations}
\subsection{Incorrect English grammar}
project who uses Solr.
\subsection{Name suggestions}
Cannot suggest names based on two words.

\section{Limitations}
Can be applied in any context where one want to instruct a computer by using a natural language.
\section{Related work}
\section{Future work}
\subsection{Speech to text}
\subsection{Proper handling of ambiguous instructions}
\subsection{Prediction function in PGF in combination with "previous words"}
This will also make it possible to suggest infinitely many suggestions in recursive functions as we can suggest based on a sentence and continue to suggest on the resulting sentence.
\subsection{Generate Solr-queries to store in the index}
Map each English or Swedish instruction to the corresponding Solr instruction, then no need of translating with GF anymore.