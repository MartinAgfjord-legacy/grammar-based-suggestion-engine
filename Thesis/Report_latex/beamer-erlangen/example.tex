\documentclass{beamer}

\usepackage{color}
\definecolor{ForestGreen}{RGB}{60, 128, 49}
\definecolor{Blue}{RGB}{45, 47, 146}

\usetheme[menuwidth={0.3\paperwidth}]{erlangen}
\setbeamercovered{transparent=20}

\begin{document}

\title[Interpretation of natural language instructions]{Interpretation of natural language instructions} 
\subtitle{Translating sentences by using a grammar}
\author{Martin Agfjord} 
\date{\today} 
\institute{University of Gothenburg\\ 
  \vspace{-0.2em}{\small Computer Science  and Engineering} }

\begin{frame}[plain]
    \thispagestyle{empty}

  \begin{picture}(0,0)(0,0)
    \put(-29,-171){\includegraphics[width=1.01\paperwidth]{images/titlepage2.pdf}}
  \end{picture}

  \begin{picture}(0,0)(0,0)
    \put(10,0){
        \begin{minipage}{0.9\linewidth}
        \centering{\LARGE\color{erlangenwhite} \inserttitle\par}
        \vspace{5mm}
        \centering{\large\color{erlangenlightgrey} \insertsubtitle\par}
        \end{minipage}
    }
  \end{picture}
  \begin{picture}(0,0)(0,0)
    \put(10,-65){
        \begin{minipage}{0.90\linewidth}
        \centering{\large\color{erlangenwhite} \insertauthor\par}
        \end{minipage}
    }
  \end{picture}
  \begin{picture}(0,0)(0,0)
    \put(10,-127){
        \begin{minipage}{0.90\linewidth}
        \centering{\color{erlangenblue} \insertinstitute\par}
        \end{minipage}
    }
  \end{picture}
\end{frame}

\begin{frame}{Outline}
  \tableofcontents
\end{frame}

\section{Introduction \& problem description}
\begin{frame}{Introduction \& problem description} 
    \begin{itemize}
    \item An alternative user interface \pause
    \item Translation \pause % from natural language to query language
    \item Delimitation 
      \begin{itemize}
        \item Intranet of a software development company
        \item Customers, People and Projects exists
	    \item Limited amount of instructions
      \end{itemize}
  \end{itemize} 
\end{frame}

\begin{frame}{Interface definition} 
\begin{block}{Sufficient for novice users}
      \texttt{people who know Java}
\end{block}
\pause
\begin{block}{Sufficient for expert users}
      \texttt{people know java}
\end{block}
\end{frame}

\section{Solution} 
\begin{frame}{Solution} 
  \begin{itemize}
    \item Precise translation \pause
    \item Need mapping from natural language to query language \pause
      \begin{itemize}
        \item Use a grammar
      \end{itemize}
  \end{itemize}
  
\end{frame}

\begin{frame}{Translate by using a grammar} 
         \begin{itemize}
           \item Structured rules for strings \pause
           \item Use logic to combine strings in one language \pause
           \item Use the same logic to combine strings in another language \pause
           \item Grammars have a long history within programming languages\pause
         \end{itemize}
           \begin{block}{How can we build a grammar to translate sentences?}
              We will use Grammatical Framework (GF)
         \end{block}
\end{frame}

\begin{frame}{Introducing Grammatical Framework (GF)} 
         \begin{itemize}
           \item Development platform for natural languages \pause
           \begin{itemize}
              \item Open source functional programming language \pause
              \item Designed for creating natural language grammars
               %specifically designed for writing grammars.
           \end{itemize}\pause
           \item Separates abstract and concrete syntax \pause
           \begin{itemize}
              \item Abstract syntax captures the \emph{logic} of a sentence \pause
              % It is a tree!
              \item Concrete syntax represents the logic as a string
           \end{itemize}\pause
         \end{itemize}
         
           \begin{block}{Same technique used by programming languages}
             \begin{itemize}
               \item Programmer writes source code in concrete syntax \pause
               \item Compiler translates concrete syntax to abstract syntax \pause
               \item The rest of the compiler manipulates the abstract syntax
             \end{itemize}
         \end{block}
         
\end{frame}

\begin{frame}[fragile]{A simple example}
\textbf{Abstract syntax}
\begin{semiverbatim}
    Instruction
   /           \\
People        Know
                |
              Java
\end{semiverbatim}\pause

\textbf{Concrete syntaxes}
\begin{semiverbatim}
people who know Java                           -- English
personer som kan Java                          -- Swedish
q=object_type : Person AND expertise : Java    -- Solr
\end{semiverbatim}
\end{frame}

\begin{frame}[fragile]{GF implementation: Abstract syntax}
\begin{semiverbatim}
abstract Instrucs = \{  
  cat \textcolor{ForestGreen}{
    Instruction 
    Subject ;
    Relation ;
    Object ; }
  fun \textcolor{Blue}{
    MkInstruction : Subject -> Relation -> Instruction ;
    People : Subject ;
    Know : Object -> Relation ; }
\}
\end{semiverbatim}
\end{frame}

\begin{frame}[fragile]{GF implementation: English concrete syntax}
\begin{semiverbatim}
concrete InstrucsEng of Instrucs = \{
  lincat \textcolor{ForestGreen}{
    Instruction = Str ;
    Subject = Str ;
    Relation = Str ;
    Object = Str ; }
  lin \textcolor{Blue}{
    MkInstruction subject relation = 
                   subject ++ "who" ++ relation ;
    People = "people" ;
    Know object = "know" ++ object ;
    Java = "Java" ; }
\}
\end{semiverbatim}
\end{frame}

\begin{frame}[fragile]{GF implementation: Solr concrete syntax}
\begin{semiverbatim}
concrete InstrucsEng of Instrucs = \{
  lincat \textcolor{ForestGreen}{
    Instruction = Str ;
    Subject = Str ;
    Relation = Str ;
    Object = Str ; }
  lin \textcolor{Blue}{
    MkInstruction subject relation = 
                   "q=" ++ subject ++ "AND" ++ relation ;
    People = "object_type : Person" ;
    Know object = object = "expertise : " ++ object ;
    Java = "Java" ; }
\}
\end{semiverbatim}
\end{frame}

\begin{frame}[fragile]{GF implementation: Translation}
GF + Abstract syntax + Concrete syntax = \pause
\begin{itemize}
\item Parser\pause
\item Linearizer\pause
\item Generator
\end{itemize}
\begin{semiverbatim}

\end{semiverbatim}
\end{frame}

\section{Results} 
\begin{frame}{Results} 
foobar
\end{frame}

\section{Conclusion} 
\begin{frame}{Title} 
  This is an example.
\end{frame}



\subsection{Enumerate}
\begin{frame}{Enumerate}
\begin{columns}
  \begin{column}{0.49\paperwidth}
      \begin{enumerate}
        \item Lorem ;
        \item<2-> Ipsum ;
        \item<2-> Dolor ;
        \item<2-> Sit amet.
      \end{enumerate}
  \end{column}

  \begin{column}{0.49\paperwidth}
    Lorem ipsum dolor sit amet, consectetur adipiscing elit. Curabitur 
    ultricies, dui ac luctus pellentesque, nunc dui lobortis lorem, nec 
    porta mauris massa vel ante. Maecenas justo nisl, sodales quis 
    placerat nec, convallis mollis magna. Curabitur blandit elementum 
    elit et euismod. In et purus nisl, ac elementum orci.
  \end{column}
\end{columns}

\end{frame}

\section{Blocs}
%%%%%%%%%%%%%%%%%%%%%%%%%%%%%%%%%%%%%%%%%%%%%%%%%%%%%%%%%%%%%%%%%%%%%%%%%%%%%%%%%%%%%%%%%%%%%%%%%%%%%%%%%%%%%%%%%%%%%%%%
\begin{frame}{}
    \sectionframe{Blocks \\ \textit{Lorem ipsum dolor sit amet}}
\end{frame}

\begin{frame}{Block}
  \begin{block}{Aliquam quis eros}

    Aliquam quis eros nec risus mattis porttitor ac eget justo. 
    Aliquam lacinia condimentum tempus. Nulla metus magna, feugiat 
    id faucibus ut, commodo et justo. Aliquam tincidunt purus vitae 
    lacus dictum in placerat purus mollis. 

  \end{block}
\end{frame}

\begin{frame}{Example block}
  \begin{exampleblock}{Nulla cursus vehicula cursus}


    Nulla cursus vehicula cursus. Suspendisse ac enim eget purus 
    tincidunt eleifend et et purus. Suspendisse ultricies viverra 
    sodales. Proin non congue risus. Maecenas vel ornare sem. Ut 
    laoreet nibh tempor felis suscipit dignissim. Aliquam urna 
    eros, sagittis interdum ultrices sed, venenatis eu nisl.

  \end{exampleblock}
\end{frame}

\begin{frame}{Alert block}
  \begin{alertblock}{Quisque vehicula pretium arcu}

    Quisque vehicula pretium arcu, eget bibendum arcu iaculis sit amet. 
    Proin facilisis sollicitudin magna, et varius lorem euismod ornare. 
    Fusce lobortis dignissim tempor. Mauris augue tortor, elementum non 
    lobortis vel, interdum vitae augue.

  \end{alertblock}
\end{frame}

\end{document}
