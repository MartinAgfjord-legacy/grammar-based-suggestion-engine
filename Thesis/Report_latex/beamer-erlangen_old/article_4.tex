%%%%%%%%%%%%%%%%%%%%%%%%%%%%%%%%%%%%%%%%%%%%%%%%
% Arsclassica Article
% LaTeX Template
% Version 1.1 (10/6/14)
%
% This template has been downloaded from:
% http://www.LaTeXTemplates.com
%
% Original author:
% Lorenzo Pantieri (http://www.lorenzopantieri.net) with extensive modifications by:
% Vel (vel@latextemplates.com)
%
% License:
% CC BY-NC-SA 3.0 (http://creativecommons.org/licenses/by-nc-sa/3.0/)
%
%%%%%%%%%%%%%%%%%%%%%%%%%%%%%%%%%%%%%%%%%%%%%%%%

%----------------------------------------------------------------------------------------
%	PACKAGES AND OTHER DOCUMENT CONFIGURATIONS
%----------------------------------------------------------------------------------------

\documentclass[
10pt, % Main document font size
a4paper, % Paper type, use 'letterpaper' for US Letter paper
oneside, % One page layout (no page indentation)
%twoside, % Two page layout (page indentation for binding and different headers)
headinclude,footinclude, % Extra spacing for the header and footer
BCOR5mm, % Binding correction
]{scrartcl}

\input{structure.tex} % Include the structure.tex file which specified the document structure and layout

\hyphenation{Fortran hy-phen-ation} % Specify custom hyphenation points in words with dashes where you would like hyphenation to occur, or alternatively, don't put any dashes in a word to stop hyphenation altogether

%----------------------------------------------------------------------------------------
%	TITLE AND AUTHOR(S)
%----------------------------------------------------------------------------------------

\title{\normalfont\spacedallcaps{Presentation content}} % The article title

\author{\spacedlowsmallcaps{Martin Agfjord}} % The article author(s) - author affiliations need to be specified in the AUTHOR AFFILIATIONS block

\date{} % An optional date to appear under the author(s)

%----------------------------------------------------------------------------------------

\begin{document}

%----------------------------------------------------------------------------------------
%	HEADERS
%----------------------------------------------------------------------------------------

\renewcommand{\sectionmark}[1]{\markright{\spacedlowsmallcaps{#1}}} % The header for all pages (oneside) or for even pages (twoside)
%\renewcommand{\subsectionmark}[1]{\markright{\thesubsection~#1}} % Uncomment when using the twoside option - this modifies the header on odd pages
\lehead{\mbox{\llap{\small\thepage\kern1em\color{halfgray} \vline}\color{halfgray}\hspace{0.5em}\rightmark\hfil}} % The header style

\pagestyle{scrheadings} % Enable the headers specified in this block

%----------------------------------------------------------------------------------------
%	TABLE OF CONTENTS & LISTS OF FIGURES AND TABLES
%----------------------------------------------------------------------------------------

\maketitle % Print the title/author/date block

\setcounter{tocdepth}{2} % Set the depth of the table of contents to show sections and subsections only

\tableofcontents % Print the table of contents

\newpage % Start the article content on the second page, remove this if you have a longer abstract that goes onto the second page

%----------------------------------------------------------------------------------------
%	INTRODUCTION
%----------------------------------------------------------------------------------------

\section{First page and outline}
Welcome to my masters project presentation! My name is Martin Agfjord and my project is about translating natural language sentences into machine readable instructions.
\subsection{Outline}
I'll first introduce the problem that we want to solve. Then I'll go through a solution to the problem, which is the largest part of the presentation. Then I'll show some results from the solution. Lastly, I'll go through what I can conclude from this work. I'll also talk a bit about some future work.
\section{Introduction \& problem description}
If a user wants to give a computer an instruction through a graphical user interface, it will probably use graphical elements like text fields, checkboxes and drop-down menus. In this project, we introduce an alternative interface, an interface which allows the user to write an instruction in a natural language.
\newline
\newline
This instruction is then translated into a machine readable language. We have chosen to use Solr query language as target language. The reason is because I have worked in close collaboration with a software development company called Findwise, and they frequently use Solr.
\newline
\newline
We need to have some kind of environment in order to define some instructions, therefore, we assume that we build this system for a software development company and the instructions are used to obtain information from their intranet. In fact, this project has gotten much inspiration from Findwise's intranet. This project only focus on a few instructions, because of limited time.

\subsection{Interface definition}
We assume that it is possible to obtain information of people's skills in the intranet, and the instruction \texttt{'people who know Java'} shall obtain a list of all persons that know the programming language Java. It has been said that novice users usually simply write what they want when searching on the web, 
\newline
\newline
However, expert users usually just type keywords like \texttt{people know java}. We want to design an application that is suffient for both novice and expert users.

\section{Solution}
Query languages and machine languages in general require precise syntax, which means a query cannot be interpreted if one character is in the wrong place in a query.
\newline
\newline
When translating, we want to map a natural language to Solr query language, how can this be achieved when also preserving precise syntax?
\newline
\newline
We use a grammar to solve this problem.

\subsection{Translation with a grammar}
So what is a grammar? A grammar is a set of structured rules for strings. And the rules defines whats valid to express in the language.

But what is a rule? When learning a foreign language, one usually study grammar rules. One example of a rule is the Is/Are rule in English.
\newline
\newline
 It describes that if refeering to a group of objects, one shall use \texttt{are}. If one refeers to a single object, one shall use is. It is therefore valid in english to say \texttt{The students are here}, it is however not valid to say \texttt{The students is here}.
\newline
\newline
However, in Swedish we use the word \texttt{är} in both occations. So how can we translate this sentence into english and still have a valid result?
\newline
\newline
We use the rules to analyze the string and obtain the logic behind it. We represent the result in a general way. In this example we see that the grammar used the rules to obtain that the word studenterna is in definite and plural form.
\newline
\newline
And we use the logic we obtained to produce a sentence in another language.
\newline
\newline
So how can we build a grammar to translate sentences? We will use Grammatical framework.

\subsection{Introducing Grammatical Framework (GF)}
GF is a development platform for natural languages.
\newline
\newline
Grammars that can be used with GF are written in the GF programming language, which is a functional programming language.
\newline
\newline
This language is specifically designed for creating natural language grammars.
\newline
\newline
One of GFs powers is that it separates abstract and concrete syntax in the same sense as programming languages.
\newline
\newline
The abstract syntax captures the logic of a sentence. Or in other words, it captures the meaning of a sentence. This is achieved by using a hieratical tree structure.
\newline
\newline
The concrete syntax expresses how an abstract syntax tree looks like a string.
\newline
\newline
Compilers also uses abstract and concrete syntax.
\newline
\newline
The programmer writes source code in concrete syntax
\newline
\newline
The compiler translates the concrete syntax into an abstract syntax tree
\newline
\newline
The compiler can then manipulate the abstract syntax tree.

\subsection{A simple example}
Here we can see an example of an abstract syntax tree which I've developed for this presentation. This abstract syntax tree captures the meaning of the sentence \texttt{people who know Java}. This is my way of representing it, it can be represented in many other different ways.
\newline
\newline
We also have three concrete syntaxes that represents this abstract syntax as strings. The first two are natural languages, while the third is a representation in Solr query language in our environment.

In order to represent this abstract syntax in a concrete syntax, one combines the function \texttt{People} with the function \texttt{Know}, however the abstract syntax leaves the implementation to the concrete syntax, so two concrete syntaxes can therefore implement the same function in different ways. This can be seen by looking at the concrete syntaxes. While we don't know how these concrete syntaxes are implemented, a reasonable assumption is that the function \texttt{People} is implemented as \texttt{"personer"} in Swedish and as \texttt{"people"} in English.

\subsection{GF implementation: Abstract syntax}
I will now show how the example I just described can be implemented in GF. When developing a grammar in GF, one always starts with the abstract syntax to define the logic.
\newline
\newline
We start by defining our categories here in green. A category in GF is the same as a data type.
\newline
\newline
We then proceed to define how one can create an instruction. An instruction is created by executing the function \texttt{MkInstruction} with two arguments, one of the type \texttt{Subject} and one of the type \texttt{Relation}. The function can then output a value of the type Instruction.

A value of the type subject can be obtained by executing the function \texttt{People} with no arguments. A value of the type \texttt{Relation} can be obtained by executing the function \texttt{Know} with an argument of the type \texttt{Object}. Lastly, a value of the type \texttt{Object} can be obtained by executing the function \texttt{Java} with no arguments.

\subsection{GF implementation: English concrete syntax}
We will now implement a concrete syntax for English. We define all categories to be strings. The function implementations is done by concatenating strings. In MkInstruction we concatenate the subject argument with the string \texttt{who}, and we then concatenate the resulting string with the relation argument.

The function \texttt{People} is implemented as the string "people". Know is implemented by concatenating the string "know" with the object argument. Java is implemented as the string "Java".

\subsection{GF implementation: Solr concrete syntax}
We will also show how one can implement the concrete syntax for Solr query language. Similarly as with the concrete syntax for English we define all categories to be strings.
\newline
\newline
The functions are also concatenation of strings, but the strings are mostly different. To create an instruction we again use MkInstruction and concatenate subject with \texttt{q=} to form a Query parameter. A Solr query contains of several boolean statements, and we assume that the arguments subject and relation are valid boolean statements, we den add an \texttt{AND} between them.

In the function People we say that the field \texttt{'object\_type'} shall have the value \texttt{'Person'}. 

The function Know is implemented similarly, but the value is dynamic as we concatenate the object argument.

Java is just the string \texttt{Java}

\subsection{GF implementation: Translation}
So now when we have our abstract syntax and our concrete syntaxes, what can we do with them?
\newline
\newline
We can parse strings from concrete syntax to abstract syntax by using the parser that GF created.
\newline
\newline
We can go from abstract syntax to concrete syntax. We say that we linearize an abstract syntax tree to concrete syntax. By using both the parser and the linearizer, one can translate a string in one concrete syntax into another string in another concrete syntax.
\newline
\newline
It is also possible to generate all possible abstract syntax trees with the generator. However, as our grammar is very small, only one abstract syntax tree exists.

\subsection{GF implementation: Resource Grammar Library}
I am now going to talk about the GF Resourcec Grammar Library which is a library that one can use when developing a concrete syntax for a natural language.
\newline
\newline
The library contains the morphology and basic syntax of currently 29 languages. 
\newline
\newline
I.e, it contains types for nouns, verbs, adjectives, noun phrases, verb phrases, relative sentences, phrases and so on. 
\newline
\newline
by using these functions, a programmer does not need to know the linguistics of each natural language and does therefore not need to focus on tedious low-level details like in a natural language for example word order or agreement.
\newline
\newline
For example, consider the sentence \texttt{'Yesterday I ate an apple'}. If we just translated each word into swedish we would get the sentende \texttt{'Igår jag åt ett äpple'} but this looks odd right?
\newline
\newline
The correct translation is \texttt{'Igår åt jag ett äpple'}.
\newline
\newline
By using GF resource grammar library, one only needs to know the domain specific words in order to translate grammatically correct sentences. In this case one need so know the words Yesterday, I, ate and apple in Swedish. The programmer can then use these words with the functions that exists in the resource grammar library.

\subsection{Resource Grammar Library: English concrete syntax}
I will now show how English for our grammar can be implemented by using the Resource Grammar Library.
\newline
\newline
We'll start by defining our categories. We use types from the Resource Grammar Library, an instruction is a noun phrase, a subject is a noun, a relation is a relative sentence and an object is a noun phrase.
\newline
\newline
The functions now use the functions provided by the resource grammar library. I'll start explaining the simplest functions. 

Java is a proper name which is converted into a noun phrase. 

The function \texttt{People} is implemented as a noun with the word \texttt{person} in singular form and the word \texttt{people} in plural form.

The function Know first makes a verb out of the string \texttt{know} and then convertes the verb into a verb second value. We then combine the verb with the object by creating a verb phrase. We then combine the verb phrase and the relative pronoun \texttt{which\_RP} by creating a Relative Clause. Finally, we can convert the relative clause into a relative sentence.

An instruction is created by combining the subject which is a noun with the relation which is a relative sentence and creating a common noun. In order to only allow plural forms of the subject noun we use a combine a plural determiner together with our common noun and creates a noun phrase.

\subsection{Resource Grammar Library: Swedish concrete syntax}
I've also included resource grammar implementation for Swedish, where I've just exchanged the strings into Swedish words. So once you have the concrete syntax for one language, it is fairly easy to develop concrete syntaxes for other languages.

\subsection{Extending the grammar}
I'll now show I've extended my application to be sufficient to my requirements.

\subsection{Extending the grammar: All programming languages}
The grammar we previously developed could only translate one sentence regarding the programming language \texttt{Java}. We would like the grammar to support \emph{any} programming language. 
\newline
\newline
We solve this by using arbitrary names instead of hard coded functions.
\newline
\newline
Remember how we defined the function Java to be of the type Object and linearized into the string \texttt{"Java"}.
\newline
\newline
We replace this function with the function \texttt{MkObject} which make use of the built in type \texttt{Symb}. Symb is a special data type which has a field named s which contains a string. GF will now understand that the function \texttt{MkObject} can take any on any value.

\subsection{Extending the grammar: More instructions}
Even though we just introduced arbitrary names, the grammar is still only supports one kind instruction.
\newline
\newline
We want to extend the grammar to support additional valid instructions regarding customers, people and projects.
\newline
\newline
These are the instructions that the final application shall support.
\newline
\newline
However, we do not want to support invalid sentences like \texttt{'projects who work in London'}.

\subsection{Extending the grammar: More instructions}
We resolve this problem by adding more categories.
\newline
\newline
And we assign people to the Internal category, customer to the external category and project to the resource category.
\newline
\newline
The relation functions are also changed, Know returns a value of the type InternalRelation and we Use function for externals returns a value of the type ExternalRelation and Use function for resources returns a value of the type ResourceRelation. I have omitted the implementation of the other relation functions due to lack of space in this slide.
\newline
\newline
Instructions can be created by combining a two values of the correct types. For example a value of the type internal with a value of the type InternalRelation, which in this case can be obtained by executing the functions \texttt{People} and \texttt{Know}.

\subsection{Extending the grammar: Boolean operators}
It is very common in query languages to use boolean operators in between statements, we would like our grammar to support translation of such instructions from a natural language to query language.
\newline
\newline
For example, the instruction \texttt{'people who know Java and Python'} shall retrieve all persons in the database that know the programming languages Java and Python.
\newline
\newline
Another example is the instruction \texttt{people who know Java or work in London}.
\newline
\newline
We add support for these kind of instructions by adding recursive functions, one for \texttt{And} and one for \texttt{Or}. The implementation here shows how its done in the concatenation version of the concrete syntax for English, where we simply add the boolean operator in between the two objects.

\subsection{Suggestion Engine}
We have developed a grammar which can translate a few sentences into Solr query language. However, we have built a \emph{narrow application grammar}, which means that it requires precise input. But how can a user translate anything if she have no idea of what instructions the application supports? And how can we support the use of only keywords when translating sentences?
\newline
\newline
We need a mapping from invalid or partial instructions into supported instructions.
\newline
\newline
We use a suggestion engine to accomplish this.
\newline
\newline
As GF can generate all possible instructions, we can store the instructions in a Solr index which we can use to map an invalid or partial instruction into an existing one.
\newline
\newline
However, as GF has now idea of what kind of names the application shall support we run into problems.
\newline
\newline
If we generate abstract syntax trees, we'll see that GF use the name \texttt{'Foo'} for all arbitrary names.
\newline
\newline
And if we linearize them into English, we'll get instructions like \texttt{'people who know Foo'}. If we store these sentences in the solr index, we will not obtain any result if we just type a name, like \texttt{Java, Python} or any other name that exists in the database.

\subsection{Suggestion Engine 2}
We resolve this by introducing types of names.

We introduce the types. Skill, Organization, Module and Location. Each of them will be used together with the appropriate relation function.
\newline
\newline
If we now generate abstract syntax trees again, we will see that each name is still named \texttt{'Foo'}, but it is of a different type, here in purple color.
\newline
\newline
We postprocess these trees and set the each name to be the corresponding type.
\newline
\newline
If we now linearize these trees into concrete syntax, we'll receive something better. The sentence \texttt{people who know Skill0} now dictates that Skill0 can be replaced by a name of the type Skill. We store all linearizations in the Solr index. Note that we still don't have any actual valid names in the index, so a search for a name like \texttt{'Java'} not give any result at all. 

\subsection{Suggestion Engine: Pseudocode of algorithm}
I'll now explain how we can use the Solr index we just created to give suggestions.
\newline
\newline
If we type the sentence \texttt{"anyone that ever knew java"} we first extract all names from the sentence. This is achieved by having a separate index with all supported names with corresponding types.
\newline
\newline
We then replace each name in the sentence with its type followed by a number.
\newline
\newline
We then query the solr index with the linearizations and we'll retrieve a list of suggested instructions.
\newline
\newline
Finally, we change back each type into the correct name.

\section{Results}
I will now show the actual results from the application and as I like applications much more than I like charts
\newline
\newline
I'm going to give the reults live by using the web application.

\section{Conclusion}
From the results we can conclude that the application is sufficient for both novice and experts users as it supports suggestions based on both partial sentences and keywords.
\newline
\newline
While I find the resource grammar library very useful for translating between two natural languages, I did not think it was that useful when translating from a natural language to a query language. Especially not if you want to map a lot of grammatically incorrect sentences into valid queries.
\newline
\newline
It is also not possible to express the sentence \texttt{'people that know Java'}, we could however express \texttt{people who know Java}.
\newline
\newline
Also, this sentence could not be expressed by using the resource grammar library.
\newline
\newline
We could however express this very similar sentence.
\newline
\newline
We also had a problem with the constant \texttt{which\_RP} which does not linearize correctly in some contexts.
\newline
\newline
Even though \texttt{'projects'} is a noun in plural form, \texttt{who} shall only be used if the noun a group of humans, like people.

\subsection{Future Work}
While the suggestions work fairly good, theres always room for improvements.
\newline
\newline
It would be good to suggest instructions even though the user hasn't typed anything.
\newline
\newline
If a user starts to type a valid sentence without any name, the application will fetch names to fill in the missing spots in the suggestions. It would be good to add some heuristic so we could fetch the most relevant names.
\newline
\newline
The application cannot translate invalid instructions at the moment, it just suggests valid sentences based on the input. It would be good take the first suggestion if the sentence is invalid.
\newline
\newline
There exists various speech recognition software today, and some of them are even free. It would be nice to use such software to transform speech into text. We can then use the resulting string with the application in order to add speech to instruction feature to the application.
\newline
\newline
When a user types an ambiguous instruction, the application just returnes the resulting abstract syntax trees. It would be good to explain to the user that the instruction was ambiguous and ask the user to clarify the instruction.
\newline
\newline
I don't know how realistic my example have been in the application with the intranet of a software development company, but I do think it can be used in many other environments that can be conrolled by natural language input. One does not have to be limited to translation from natural language to query language, it is very easy to use another machine language. I do have some ideas of my own that I'm going to experiment with after this.
\end{document}