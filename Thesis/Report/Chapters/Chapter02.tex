\chapter{A simple grammar}\label{ch:simple-grammar}
This chapter presents an example of how GF can be used to create a grammar that can translate the sentence \emph{people who know Java} into Apache Solr query language and vice verca. Apache Solr is a search platform based on Apache Lucene \cite{kuc2011apache}.

\section{Abstract syntax}
To model the meaning of a sentence, GF adopts the use of functions and \emph{categories}. A category (\texttt{cat}) in GF is the same as a data type. We start by listing the categories we need, as seen in \autoref{fig:abstract-syntax} on lines 3-7. We then define the values that our data types can take. This is achieved by using functions. The functions in an abstract syntax are usually not implemented, we can therefore only see the function declarations. The reason is because we only want to model the semantics at the abstract level. How the semantics are implemented in a specific language is irrelevant, because we want to keep the abstract syntax as language independent as possible in order to make it easier to develop concrete syntaxes.

We define a function \texttt{Java : Object} which means that \texttt{Java} is a constant \marginpar{A function without arguments is called a constant in lazy functional programming languages.} and returns a value of type \texttt{Object}. \texttt{Know} takes one argument of the type \texttt{Object} and returns a value of type \texttt{Relation}.

An instruction can be created by obtaining a value of the type \texttt{Instruction}. Only \texttt{MkInstruction} returns the desired type and it takes two arguments, one of type \texttt{Subject} the other of type \texttt{Relation}.

\begin{figure}[H]
\begin{code}
abstract Instrucs = { 
  flags startcat = Instruction; 
  cat
    Instruction ; -- An Instruction
    Subject ;     -- The subject of an instruction
    Relation ;    -- A verb phrase
    Object ;      -- an object

  fun		 	  
    MkInstruction : Subject -> Relation -> Instruction ;
    People : Subject ;
    Know : Object -> Relation ;
    Java : Object ; 
}
\end{code}
\caption{Abstract syntax\label{fig:abstract-syntax}}
\end{figure}

%We can now use this abstract syntax to create an abstract syntax tree as seen in \autoref{fig:ast}

%\begin{figure}[h]
%\begin{plaintext}
%MkInstruction People (Know Java)
%\end{plaintext}
%\caption{Abstract syntax tree\label{fig:ast}}
%\end{figure}

\section{Concrete syntax}

We are now going to implement the linearizations of the function declarations we just defined in the abstract syntax. This implementation makes it possible to linearize abstract syntax trees into concrete syntax. We will start by defining the concrete syntax for English.

\hiddensubsection{English concrete syntax}

\autoref{fig:english-concr} shows the implementation of the concrete syntax for English. Categories are linearized by the keyword \texttt{lincat}, which literally means the linearization of categories. A category is linearized by assigning a data type to it. Here we assign all categories to be of the type string, which means that all our linearization functions also must return a string. The functions are linearized by using the keyword \texttt{lin}. We linearize \texttt{Java} by returning the string \texttt{"Java"}, as it is a constant function. Analogously, \texttt{"people"} is returned by \texttt{People}. The function \texttt{Know} takes one parameter. This parameter is appended on the string \texttt{"know"}. Finally, \texttt{MkInstruction} takes two arguments, where \texttt{subject} is prepended and \texttt{relation} is appended on \texttt{"who"}. One can easily see how these functions can be used to construct the sentence \emph{people who know Java}.

\newenvironment{myfont}{\myfont}{\par}

\begin{figure}[H]
\begin{code}
concrete InstrucsEng of Instrucs = {
  lincat
    Instruction = Str ;
    Subject = Str ;
    Relation = Str ;
    Object = Str ;
  lin
    MkInstruction subject relation = subject ++ "who" ++ relation ;
    People = "people" ;
    Know object = "know" ++ object ;
    Java = "Java" ;
}
\end{code}
\caption{English concrete syntax\label{fig:english-concr}}
\end{figure}

\hiddensubsection{Solr concrete syntax}

The final step in this example is to linearize the same abstract syntax into Solr concrete syntax.
  As \autoref{fig:solr-concr} shows, the categories are strings as in English. The function linearizations are however
  different. \texttt{People} returns 
  \texttt{"object\_type : Person"}, we assume that 
  the Solr-schema has a field with the name \texttt{object\_type} which represents 
  the type of a document. Similarly, we make another assumption about \texttt{Know}. \texttt{MkInstruction} is also implemented differently, here we can see that the result is going to be a query string\marginpar{A query string is a part of a URL, e.g. foo.com?\textbf{q=name}} by looking at the first part \texttt{"q="} which is prepended on the subject. We then append \texttt{"AND"} together with \texttt{relation} in order to create a valid Solr query.

\begin{figure}[H]
\begin{code}
concrete InstrucsSolr of Instrucs = {
    lincat
      Instruction = Str ;
      Subject = Str ;
      Relation = Str ;
      Object = Str ;

    lin
      MkInstruction subject relation = "q=" ++ subject ++ "AND" ++ relation ;
      People = "object_type : Person" ;
      Know object = "expertise : " ++ object ;
      Java = "Java" ;
}
\end{code}
\caption{Solr concrete syntax\label{fig:solr-concr}}
\end{figure}

\section{Translation}

In order to make any translations, we need to use the GF runtime system. The  runtime system we will use in this section is the shell application, which allows us to load our GF source code and use parsers, linearizers and generators. In addition to the shell application, there also exists programming libraries for GF in C, Haskell, Java and Python. These libraries can be used to build a translation application which does not require the user to have GF installed.

\begin{figure}[H]
\begin{terminal}
$ gf InstrucsEng.gf InstrucsSolr.gf 
                                
       *  *  *              
    *           *           
  *               *         
 *                          
 *                          
 *        * * * * * *       
 *        *         *       
  *       * * * *  *        
    *     *      *          
       *  *  *              
                            
This is GF version 3.5.12-darcs. 
No detailed version info available
Built on linux/x86_64 with ghc-7.6, flags: interrupt server
License: see help -license.   
Bug reports: http://code.google.com/p/grammatical-framework/issues/list
 - compiling Instrucs.gf...   write file Instrucs.gfo
- compiling InstrucsEng.gf...   write file InstrucsEng.gfo
- compiling InstrucsSolr.gf...   write file InstrucsSolr.gfo
linking ... OK
 Languages: InstrucsEng InstrucsSolr
Instrucs> 
\end{terminal}
%$ <- Texmaker sux at syntax highlighting

\caption{GF shell prompt\label{fig:gf-shell}}
\end{figure}

A string can be parsed into an abstract syntax tree.

\begin{figure}[H]
\begin{terminal}
Instrucs> parse -lang=InstrucsEng "people who know Java"
MkInstruction People (Know Java)
\end{terminal}

\caption{Parse a string\label{fig:parse-a-string}}
\end{figure}

Abstract syntax trees can be linearized into concrete syntaxes, here we linearize one abstract syntax tree into all known concrete syntaxes.

\begin{figure}[H]
\begin{terminal}
Instrucs> linearize MkInstruction People (Know Java)
people who know Java
q= object_type : Person AND expertise : Java
\end{terminal}

\caption{Linearize an abstract syntax tree\label{fig:linearization}}
\end{figure}

Finally, a string can be translated from one concrete syntax into another. Here we translate from \texttt{InstrucsEng} into \texttt{InstrucsSolr}. We use a \emph{pipeline} to pass the result of the parsing as an argument to the linearizing function. Note how we use \texttt{p} instead of \texttt{parse} and \texttt{l} instead of \texttt{linearize}. They are just shorthands of their longer representations.

\begin{figure}[H]
\begin{terminal}
Instrucs> p -lang=InstrucsEng "people who know Java" | l -lang=InstrucsSolr
q= object_type : Person AND expertise : Java
\end{terminal}
\caption{Translate between concrete syntaxes\label{fig:translation-concr}}
\end{figure}

\section{GF resource grammar library}

The previous example is fairly easy to understand, but it also make use of English, a well-known natural language. It is much harder to create a concrete syntax that implements a lesser known natural language by using concatenation. Even though a user might know correct translation of the individual words to use, she might not know how to use them in a grammatically correct sentence. It is often the case that if one directly translates a sentence, i.e. just translate each word by word, one ends up in a grammatically incorrect sentence.
\newline
\newline
The GF Resource Grammar Library (RGL) \cite{gf-rgl} is a set of grammars which implements the morphology and basic syntax of currently 29 languages \cite{gf-rgl-synopsis}. In other words, it contains functions and categories which describes the structure of natural languages. One can therefore create values of specific types from the words of a sentence and then combine the words by using functions in order to create a grammatically correct phrase or sentence. We say that a developer does only need to have knowledge of her \emph{domain}, i.e. the individual words to use, and does not have any specific linguistic knowledge of the natural language.

\subsection*{Example usage of the RGL in a grammar}

In this section, we will present how the previous concrete syntax for English can be implemented by using the RGL. We will also show how this grammar can be further generalized into an incomplete concrete syntax which can be used by both English and Swedish.

\autoref{fig:english-RGL} shows the concrete syntax for English by using the RGL. The categories are now set to be types that exists in the RGL and the functions are now using RGL-functions in order to create values of the correct types.

The most simple function in this case is \texttt{People}, which shall return a noun. A noun can be created by using the \emph{operation}\marginpar{An operation in GF is a function which can be called by linearization functions.} \texttt{mkN}. We create a noun which has the singular form \texttt{"person"} and plural form \texttt{"people"}, we will never use the singular form in this grammar, but it will become handy later in the thesis to use both singular and plural forms.
\newline
\newline
\texttt{Java} returns a \texttt{noun phrase} which is created by the function \texttt{mkNP}, however, we create a noun phrase by converting a proper name which is initialized as \emph{Java}. \texttt{Know} returns a relative sentence. A relative sentence can for example be \emph{who know Java}. A relative sentence is constructed by first creating a \emph{verb phrase} from a verb and an object. This verb phrase is then used together with the constant \texttt{which\_RP} to create a \emph{relative clause}. Finally, we convert the relative clause into a relative sentence. This is achieved by using a self made operation named \texttt{mkRS'}, the purpose of this operation is to make the code easier to read and also in the future reuse code. 

\begin{figure}[H]
\begin{code}
concrete InstrucsEng of Instrucs = open SyntaxEng, ParadigmsEng in {
  lincat
    Instruction = Utt ;
    Subject = N ;
    Relation = RS ;
    Object = NP ;

  lin
    MkInstruction subject relation = mkUtt 
                                      (mkNP aPl_Det (mkCN subject relation)) ;
    People = mkN "person" "people" ;
    Know object = mkRS' (mkVP (mkV2 (mkV "know") object)) ;
    Java = mkNP (mkPN "Java") ;
    
   oper
      mkRS' : VP -> RS = \vp -> mkRS (mkRCl which_RP vp) ;
}
\end{code}
\caption{English concrete syntax using the RGL\label{fig:english-RGL}}
\end{figure}

The only thing that is left is to combine a noun with a relative sentence, e.g. combine \emph{people} with \emph{who know Java}. This is done by using the operation \texttt{mkCN} to create a common noun. As common nouns do not have any determiners, we have to construct a noun phrase together with the determiner \texttt{aPl\_Det} in order to select only the plural forms. Lastly we convert the noun phrase into an utterance in order to only allow the nominative form of the sentence (we would otherwise end up with with multiple equal abstract syntax trees when parsing a sentence).


\section{Generalizing the concrete syntax}

This section describes how the concrete syntax can be generalized into an \emph{incomplete concerete syntax} and then be instantiated by two concrete syntaxes, one for English and one for Swedish.

\subsection*{An incomplete concrete syntax}

As we already have designed the concrete syntax for English, we can fairly easy convert it to a generalised version. The incomplete concrete syntax can be seen in \autoref{fig:incomplete-concrete-syntax}. We no longer have any strings defined, as we want to keep the syntax generalised. Constants are used in place of strings, and they are imported from the lexicon interface  \texttt{LexInstrucs}.

\begin{figure}[H]
\begin{code}
incomplete concrete InstrucsI of Instrucs = open Syntax, LexInstrucs in {
  lincat
    Instruction = Utt ;
    Subject = N ;
    Relation = RS ;
    Object = NP ;

  lin
    MkInstruction subject relation = mkUtt 
                                      (mkNP aPl_Det (mkCN subject relation)) ;
    People = person_N ;
    Know object = mkRS' (mkVP know_V2 object) ;
    Java = java_NP ;

   oper
      mkRS' : VP -> RS = \vp -> mkRS (mkRCl which_RP vp) ;
}
\end{code}
\caption{Incomplete concrete syntax\label{fig:incomplete-concrete-syntax}}
\end{figure}

\texttt{LexInstrucs} is an \emph{interface}, which means that it only provides declarations. \autoref{fig:lexicon-interface} shows that we have one operation declaration for each word we want to use in the concrete syntax. Because we do not implement the operations, it is possible to create multiple instances of the lexicon where each one can implement the lexicon differently.

\begin{figure}[H]
\begin{code}
interface LexInstrucs = open Syntax in {
   oper
     person_N : N ;
     know_V2  : V2 ;
     java_NP  : NP ;
}
\end{code}
\caption{Lexicon interface\label{fig:lexicon-interface}}
\end{figure}

\autoref{fig:english-lexicon} shows how the operations defined in \texttt{LexInstrucs} are implemented in \texttt{LexInstrucsEng}. We represent the words in the same  way as in the old version of the concrete syntax for English.

\begin{figure}[H]
\begin{code}
instance LexInstrucsEng of LexInstrucs = open SyntaxEng, ParadigmsEng in {
    oper
      person_N = mkN "person" "people" ;
      know_V2 = mkV2 (mkV "know") ;	  
      java_NP = mkNP (mkPN "Java");
}
\end{code}
\caption{Lexicon instantiation of English \label{fig:english-lexicon}}
\end{figure}

\autoref{fig:swedish-lexicon} shows another instance of \texttt{LexInstrucs}, the lexicon for Swedish. \marginpar{The definition of \texttt{know\_V2} is taken from StructuralSwe.gf in the RGL}

\begin{figure}[H]
\begin{code}
instance LexInstrucsSwe of LexInstrucs = open SyntaxSwe, ParadigmsSwe in {
    oper
      person_N = mkN "person" "personer" ;
      know_V2 = mkV2 (mkV "kunna" "kan" "kunn" "kunde" "kunnat" "kunnen") ;	  
      java_NP = mkNP (mkPN "Java");
}
\end{code}
\caption{Lexicon instantiation of Swedish\label{fig:swedish-lexicon}}
\end{figure}
%\todo{Why can't we define \texttt{know\_V2} as just \texttt{mkV2 (mkV "kan")}?}

We are now ready to instantiate the incomplete concrete syntax. The code below describes how \texttt{InstrucsI} is instantiated as \texttt{InstrucsEng}. Note how we override \texttt{Syntax} with \texttt{SyntaxEng} and \texttt{LexInstrucs} with \texttt{LexInstrucsEng}.

\begin{figure}[H]
\begin{code}
concrete InstrucsEng of Instrucs = InstrucsI with 
                                         (Syntax = SyntaxEng), 
                                         (LexInstrucs = LexInstrucsEng) 
                                         ** open ParadigmsEng in {}
\end{code}
\caption{English instantiation of the incomplete concrete syntax}
\end{figure}

Analogously, we create an instance for Swedish concrete syntax by instantiating \texttt{InstrucsI} and overriding with different files.

\begin{figure}[H]
\begin{code}
concrete InstrucsSwe of Instrucs = InstrucsI with 
                                             (Syntax = SyntaxSwe), 
                                             (LexInstrucs = LexInstrucsSwe) 
                                             ** open ParadigmsSwe in {}
\end{code}
\caption{Swedish instantiation of the incomplete concrete syntax}
\end{figure}

If we load the GF-shell with \texttt{InstrucsEng.gf} and \texttt{InstrucsSwe.gf} we can make the following translation from English to Swedish.

\begin{figure}[H]
\begin{code}
Instrucs> p -lang=InstrucsEng "people who know Java" | l -lang=InstrucsSwe
personer som kan Java
\end{code}
\caption{Swedish instantiation of the incomplete concrete syntax}
\end{figure}
%\todo{Why we get three results is unknown at the moment.}
Whats really interesting is that we now can go from both English and Swedish into abstract syntax, and by extension, also to Solr-syntax.