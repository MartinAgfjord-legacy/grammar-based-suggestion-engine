\documentclass[
10pt, % Main document font size
a4paper, % Paper type, use 'letterpaper' for US Letter paper
oneside, % One page layout (no page indentation)
%twoside, % Two page layout (page indentation for binding and different headers)
headinclude,footinclude, % Extra spacing for the header and footer
BCOR5mm, % Binding correction
]{scrartcl}

\input{structure.tex}

\hyphenation{Fortran hy-phen-ation}

\title{\normalfont\spacedallcaps{Destination Prediction with Decision Tree}}

\author{\spacedlowsmallcaps{An opposition by Martin Agfjord*}}

\date{} % An optional date to appear under the author(s)

\begin{document}

\renewcommand{\sectionmark}[1]{\markright{\spacedlowsmallcaps{#1}}}
\lehead{\mbox{\llap{\small\thepage\kern1em\color{halfgray} \vline}\color{halfgray}\hspace{0.5em}\rightmark\hfil}} % The header style

\pagestyle{scrheadings} % Enable the headers specified in this block

\maketitle % Print the title/author/date block

{\let\thefootnote\relax\footnotetext{* \textit{Master thesis student at Department of Computer Science \& Engineering, University of Gothenburg}}

\section*{Overall}
Presentation -
\newline
Report - Well written, Interesting problem and solution. Many positive effects of using Decision Tree over other approaches.

\section*{Evaluation}
Title - Clear and well describing

\section*{Recommendations}
\subsection*{Layout}
\subsubsection*{3.1 Problem Description}
About section \emph{3.1 Problem Description}. You describe the problem you want to solve which is crucial for a thesis. I think that this section does not belong in a chapter about \emph{Design and Implementation}. My recommendation is to move the section to the introduction. The section will then be a better hook to catch an interested reader early.
\newline
An alternative approach is change the section name to \emph{Requirements specification} or \emph{Design summary} or similar.
\subsection*{Decision Tree}
Include section about how machine learning works in general and describe the machine learning terms you are using in the thesis, e.g. (statistical) classification, train-data, test-data.
\newline
\newline
I had some problem understanding how a Decision Tree works. You describe (in related work) that one of the powers of Decision Tree is that it can be represented in a tree, and a tree can easily be visualised in order to give an example.
\newline
\newline
I also had a hard time understanding why the ID3-algorithm works, I have however not taken courses in statistics and machine learning so it might be the reason. I do recommend a more detailed description, preferably with an example.
\subsubsection*{Training- and testdata}
Overall, you describe your design and testing with data very carefully, but I miss a description of how the training and test data was decided. Without this description, one can make the very bad assumption that you tested on the same data as you trained on which would yield unreliable results.
\section*{Details}
\subsection*{Typos}
\subsubsection*{Introduction}
Page 1. helps from Ericsson -> help from Ericsson
\subsubsection*{Related Work}
Page 3. faster than human -> faster than a human
\newline
Page 3. more precisely and helping avoid -> more precise and helps to avoid
\newline


\subsection*{Misunderstandings / Unclear}
\subsubsection*{Related Work}
Page 3. However, there is \textbf{rare}\marginpar{What do you mean?} study beyond car itself that can provide suggestions and guidance to drivers before they get into trouble.
\newline
Page 5. Both of these two approaches had made \textbf{appreciate}\marginpar{Wrong word?} results.
\end{document}