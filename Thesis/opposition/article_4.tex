\documentclass[
10pt, % Main document font size
a4paper, % Paper type, use 'letterpaper' for US Letter paper
oneside, % One page layout (no page indentation)
%twoside, % Two page layout (page indentation for binding and different headers)
headinclude,footinclude, % Extra spacing for the header and footer
BCOR5mm, % Binding correction
]{scrartcl}

%%%%%%%%%%%%%%%%%%%%%%%%%%%%%%%%%%%%%%%%%
% Arsclassica Article
% Structure Specification File
%
% This file has been downloaded from:
% http://www.LaTeXTemplates.com
%
% Original author:
% Lorenzo Pantieri (http://www.lorenzopantieri.net) with extensive modifications by:
% Vel (vel@latextemplates.com)
%
% License:
% CC BY-NC-SA 3.0 (http://creativecommons.org/licenses/by-nc-sa/3.0/)
%
%%%%%%%%%%%%%%%%%%%%%%%%%%%%%%%%%%%%%%%%%

%----------------------------------------------------------------------------------------
%	REQUIRED PACKAGES
%----------------------------------------------------------------------------------------

\usepackage[
nochapters, % Turn off chapters since this is an article        
beramono, % Use the Bera Mono font for monospaced text (\texttt)
eulermath,% Use the Euler font for mathematics
pdfspacing, % Makes use of pdftex’ letter spacing capabilities via the microtype package
dottedtoc % Dotted lines leading to the page numbers in the table of contents
]{classicthesis} % The layout is based on the Classic Thesis style

\usepackage{arsclassica} % Modifies the Classic Thesis package

\usepackage[T1]{fontenc} % Use 8-bit encoding that has 256 glyphs

\usepackage[utf8]{inputenc} % Required for including letters with accents

\usepackage{graphicx} % Required for including images
\graphicspath{{Figures/}} % Set the default folder for images

\usepackage{enumitem} % Required for manipulating the whitespace between and within lists

\usepackage{lipsum} % Used for inserting dummy 'Lorem ipsum' text into the template

\usepackage{subfig} % Required for creating figures with multiple parts (subfigures)

\usepackage{amsmath,amssymb,amsthm} % For including math equations, theorems, symbols, etc

\usepackage{varioref} % More descriptive referencing

%----------------------------------------------------------------------------------------
%	THEOREM STYLES
%---------------------------------------------------------------------------------------

\theoremstyle{definition} % Define theorem styles here based on the definition style (used for definitions and examples)
\newtheorem{definition}{Definition}

\theoremstyle{plain} % Define theorem styles here based on the plain style (used for theorems, lemmas, propositions)
\newtheorem{theorem}{Theorem}

\theoremstyle{remark} % Define theorem styles here based on the remark style (used for remarks and notes)

%----------------------------------------------------------------------------------------
%	HYPERLINKS
%---------------------------------------------------------------------------------------

\hypersetup{
%draft, % Uncomment to remove all links (useful for printing in black and white)
colorlinks=true, breaklinks=true, bookmarks=true,bookmarksnumbered,
urlcolor=webbrown, linkcolor=RoyalBlue, citecolor=webgreen, % Link colors
pdftitle={}, % PDF title
pdfauthor={\textcopyright}, % PDF Author
pdfsubject={}, % PDF Subject
pdfkeywords={}, % PDF Keywords
pdfcreator={pdfLaTeX}, % PDF Creator
pdfproducer={LaTeX with hyperref and ClassicThesis} % PDF producer
}

\hyphenation{Fortran hy-phen-ation}

\title{\normalfont\spacedallcaps{Destination Prediction with Decision Tree}}

\author{\spacedlowsmallcaps{An opposition by Martin Agfjord*}}

\date{} % An optional date to appear under the author(s)

\begin{document}

\renewcommand{\sectionmark}[1]{\markright{\spacedlowsmallcaps{#1}}}
\lehead{\mbox{\llap{\small\thepage\kern1em\color{halfgray} \vline}\color{halfgray}\hspace{0.5em}\rightmark\hfil}} % The header style

\pagestyle{scrheadings} % Enable the headers specified in this block

\maketitle % Print the title/author/date block

{\let\thefootnote\relax\footnotetext{* \textit{Master thesis student at Department of Computer Science \& Engineering, University of Gothenburg}}

\section*{Overall}
Presentation -
\newline
Report - Well written, Interesting problem and solution.
I think you have solved the problem in an intuitive and elegant way by combining different services in order to give informative notifications to the user.
\section*{Evaluation}
Title - Clear and well describing

\section*{Recommendations}
\subsection*{Layout}

The thesis is divided into six chapters, Introduction, Related Work, Design and Implementation, Testing and Discussion, Conclusion and Future Work. A person with experience from technical reports like myself can easily find what it wants in the thesis because of the structure.
\newline
\newline
I do however have one recommendation about reorganizing the thesis. You introduce the concept of the connected car and describe what other companies have done previously. Then you describe what you are going to do in your thesis, but without describing the problem that you want to solve. I recommend you to include a problem description in the introduction chapter.
\newline
\newline
In fact, you already have a section about this in the chapter \emph{Design and Implementation}. This section is better suited in the introduction in order to catch an interested reader early.

\subsection*{Decision Tree}
When I read the section about Decision Tree, I had problem understanding how it works. However, after looking at the visual example presented 8 pages later, it was perfectly clear to me. I recommend you to move this example to the section where you first describe Decition Tree. Or maybe you have a specific reason why you included the visual example later?

\subsection*{ID3}
I also had some problems  understanding why the ID3-algorithm works, I had to read the section multiple times. I recommend a more detailed description, preferably with an example. I often find it easier to understand an algorithm if I can compare the text description with the algorithm in pseudocode. You include an algorithm which generates a desicion tree 6 pages later, but it is unclear wheater this is the ID3-algorithm or not. 
\newline
\newline
You describe \emph{why} the ID3-algorithm works after you described \emph{how} it works. This is the part I found most confusing, probably because I do not have much background within probability theory and machine learning. A general introduction to machine learning would be useful, where you describe how machine learning build up a model based on training data and you can describe each term by term like, training data, test data, classification, class, gain-function. 

Here is a list of recommendations:
\begin{itemize}
    \item The first thing you want to describe is that this is an algorithm which creates a decision tree. It feels like you assume the reader knows this.
    \item Include a figure with the pseudocode of ID3, and talk about how the pseudocode works, rather than describing the algorithm with only words.
    \item What is \textbf{entropy theory}? What is \textbf{disorder} in a data set? Elaborate on these two terms.
    \item What is an information-based method?
    \item I find it hard to understand \emph{why} ID3 works (page 11). There are some terms and variables used that I understand come from statistics and machine learning. Include a general introduction to machine learning and probability.
\end{itemize}

\subsubsection*{Training- and testdata}
Overall, you describe your design and testing with data very carefully, but I miss a description of how the training and test data was decided. Without this description, one can make the very bad assumption that you tested on the same data as you trained on which would yield unreliable results.

\section*{Questions}
\subsection*{Design and Implementation}
You describe that if there is logs that a driver has been starting trips at 8:30, 9:01 and 16:59, then the possible times to push notification will be 8:00, 9:00 and 16:00. Why do you have this rule? Is it not possible to send the notification 30 minutes earlier or an hour? Rather than sending the notification to the closest previous hour.

\section*{Details}
\subsection*{Typos}

\subsubsection*{Abstract}
can achieve higher 75\% -> can achieve higher than 75\%
\subsubsection*{Introduction}
Page 1. helps from Ericsson -> help from Ericsson
\subsubsection*{Related Work}
Page 3. faster than human -> faster than a human
\newline
Page 3. more precisely and helping avoid -> more precise and helps to avoid

\subsubsection*{Design and Implementation}
Page 15. recursive function recursively -> recursive function which recursively
\newline
Page 19. 9:01 and16:59 -> 9:01 and 16:59

\subsection*{Misunderstandings / Unclear}
\subsubsection*{Related Work}
Page 3. However, there is \textbf{rare}\marginpar{What do you mean?} study beyond car itself that can provide suggestions and guidance to drivers before they get into trouble.
\newline
Page 5. Both of these two approaches had made \textbf{appreciate}\marginpar{Wrong word?} results.
\end{document}